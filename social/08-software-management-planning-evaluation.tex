\documentclass{../industrial-development}
\graphicspath{{09-software-project-management/}}

\usepackage{multirow}
\newcommand{\sz}{\footnotesize}
\newcommand{\zz}{\phantom{0}}

\title{Управление программными проектами: планирование и оценка}
\author{Лагутина Ксения Владимировна, ИТ-21 МО}
\date{}

\begin{document}

\begin{frame}
  \titlepage
\end{frame}

\section{Постановка задачи управления проектом}

\subsection{Обоснование проекта, определение цели проекта, рабочей области, граничных условий}

\begin{frame} \frametitle{Процессы продукта и проекта}
  \begin{itemize}
   \item Процессы продукта описывают создание программного продукта
   \item Процессы проекта описывают, как будет выполняться разработка
  \end{itemize}
\end{frame}

\lecturenotes

Планирование проектов представляет собой процесс, который включает в себя определение задач проекта, выбор модели жизненного цикла, установку политик, процедур и процессов, необходимых для достижения целей.

Независимо от выбора жизненного цикла проекта, существует два различных набора процессов: проекта и продукта. Процессы продукта описывают, каким образом создается программный продукт. Они будут рассмотрены в следующих лекциях. Процессы проекта описывают, каким образом команда разработчиков будет определять и выполнять процессы разработки продукта~\cite[с.~241--242]{Fatrell}.

\begin{frame} \frametitle{Процессы проекта}
  \begin{itemize}
   \item Почему "--- обоснование проекта
   \item Что "--- техническое задание
   \item Как "--- модель жизненного цикла
   \item Выполнение "--- выполнение жизненного цикла
   \item Сделано "--- анализ эффективности выполнения
  \end{itemize}
\end{frame}

\lecturenotes
Схема процессов проекта состоит из пяти этапов.

Этап <<Почему>> "--- это обоснование проекта. Для каждого проекта требуется подходящая причина завершения, и на этом этапе гарантируется наличие хотя бы одной такой причины. Над обоснованием обычно работают менеджер и финансовые эксперты. В конце обоснование обязательно фиксируется и обычно становится частью технического задания.

Непосредственно после обоснования реализации программного проекта определяются цель и область действия проекта. Техническое задание представляет собой рабочий план, в рамках которого реализуется проект. В этом документе описываются разрабатываемые продукты, приведен схематический график высокого уровня, производится начальная оценка требуемых ресурсов.

Как только будет одобрено техническое задание, составляется план менеджмента программного проекта "--- этап <<Как>>. В нём объясняется порядок выполнения стадий жизненного цикла. При этом формируется своего рода <<дорожная карта>> для команды разработчиков проекта, в соответствии с которой продукт и будет разрабатываться. Детализированное планирование осуществляется на следующем шаге жизненного цикла проекта, на этапе <<Выполнение>>. Выполнение подразумевает непосредственную работу над проектом.

На последнем этапе продукт оценивается и анализируется. В ходе этого анализа можно получить рекомендации относительно текущего проекта или улучшений производственной деятельности, полезных для следующих проектов~\cite[с.~242--244]{Fatrell}.

\begin{frame} \frametitle{Определение цели}
  \begin{block}{}
    Цель должна быть четко определена!
  \end{block}
  \begin{block}{}
    Завершение проекта "--- это одна из главных целей
  \end{block}
  \begin{block}{}
    Основная цель должна быть описана сжато, но понятно для всех участников
  \end{block}
\end{frame}

\lecturenotes

При выполнении каждого проекта определяется как минимум одна цель, причем она должна быть четко определена. Определение ясной и четкой цели проекта является одним из простейших и экономных действий, осуществляемых при разработке всего проекта. Размытая цель обычно ведет к получению неопределённых результатов.

Завершение проекта "--- это обязательная цель, так как любой проект имеет начальную и конечную даты, а также ограничения по затратам, трудозатратам и качеству.

Основная цель проекта представляет собой краткое описание проекта, его особенностей и ограничений. Важно определить её в легко понимаемых и воспроизводимых терминах. На любом совещании по проекту каждый из членов команды разработчиков должен осознавать цели производимой работы. Иногда формулировку цели можно оценить с помощью так называемой беседы в лифте. Если вы за время поездки в лифте можете объяснить суть проекта постороннему человеку, то цель сформулирована хорошо~\cite[с.~244--247]{Fatrell}.

\begin{frame} \frametitle{Определение рабочей области и граничных условий}
  \begin{definition}
    Рабочая область "--- это рабочий план, включающий описание требований
  \end{definition}
  \begin{block}{}
    Необходимо точно определить границы рабочей области проекта
  \end{block}
\end{frame}

\lecturenotes

Рабочую область проекта также называют рабочим планом, описанием требований, но не детализированным. Детализированное описание требований описано в другом документе, спецификации. Рабочий план содержит описание специфичных деталей и спецификаций компонентов, которое составлено таким образом, что может быть передано субподрядчику для выполнения.

Определение граничных условий подразумевает четкое определение границ рабочей области. Во время из выявления полезно сформулировать явно два списка: то, что проект будет и не будет делать. Это позволить выявить скрытые предположения относительно проекта и занести их в спецификацию~\cite[с.~250--250]{Fatrell}.

\subsection{Составление технического задания}

\begin{frame} \frametitle{Техническое задание}
  \begin{definition}
    Техническое задание "--- это документ, формально определяющий существование проекта
  \end{definition}
  \begin{block}{Перечень вопросов по техническому заданию}
    \begin{itemize}
     \item Цели проекта
     \item Промежуточные результаты работы
     \item Контрольные точки
     \item Технические требования
     \item Ограничения и исключения
     \item Проверка выполнения работы совместно с клиентом
    \end{itemize}
  \end{block}
\end{frame}

\lecturenotes

Техническое задание (ТЗ) включает определение деловых потребностей, описание продукта, а также основные предположения. Как правило, в нем описываются высокоуровневые задачи, рабочая область проекта и другая информация высокого уровня, одобренная заказчиком. Оно обязательно включает в себя обоснование проекта и описание модели жизненного цикла. В техническом задании не описываются подробности того, что будет происходить на конкретных фазах проекта, это будет сделано позднее. Рекомендуется делать техническое задание как можно более коротким~\cite[с.~250--251]{Fatrell}.

Основное содержание ТЗ:

Цели проекта. Первым этапом в определении ТЗ является определение основных целей для удовлетворения потребностей клиента.

Промежуточные результаты работы. Следующим этапом является определение промежуточных результатов работы на протяжении всего жизненного цикла проекта. Так, например, промежуточным результатом работы на самой ранней стадии разработки проекта может быть список спецификаций. На следующем этапе это может быть испытание образцов. Последним этапом может быть окончательное испытание и одобренная программа. Промежуточные этапы работы обычно включают время, количество и/или оценки затрат.

Контрольные точки. Контрольная точка "--- это значительное мероприятие в процессе работы над проектом, которое происходит в определенный момент времени. График контрольных точек отражает только основные сегменты работы; он показывает первую, приблизительную оценку затрат времени, стоимости и необходимых ресурсов для проекта. Этот график составляется с использованием промежуточных результатов работы, как основы для определения основных сегментов работы и конечной даты.

Технические требования к продукту, в том числе спецификация его функциональности.

Ограничения и исключения. Следует четко определить границы ТЗ. Невыполнение этого требования приведет к пустым ожиданиям и трате ресурсов и времени.

Проверка выполнения работы совместно с заказчиком. Контрольный список вопросов ТЗ проекта заканчивается совместной с заказчиком проверкой выполнения работы. Основной проблемой является понимание и согласие заказчика с ожидаемыми результатами. Получает ли заказчик в виде
промежуточных результатов то, что он хочет? Указывает ли определение проекта ключевые достижения, сметы, сроки и требования к выполнению работ? Рассматриваются ли вопросы ограничений и исключений? Обсуждение всех этих вопросов крайне необходимо во избежание недопонимания.

В общем, тесное сотрудничество с вашим заказчиком необходимо для разработки такого ТЗ проекта, которое бы удовлетворяло всем требованиям заказчика. Также хорошее ТЗ будет нам необходимо, если вдруг что-то начнет меняться. Четкое определение ТЗ проекта является необходимым условием для структурирования работ по этапам. ТЗ дает административный план, который используется при разработке вашего оперативного плана. ТЗ должно быть кратким, но полным~\cite[с.~77--79]{Grey}.

\section{Оценка затрат на выполнение проекта, подходы к управлению рисками}

\subsection{Стратегии распределения времени}

\begin{frame} \frametitle{Методы распределения временных ресурсов}
    \begin{itemize}
     \item Календарный план
     \item Сетевое планирование
    \end{itemize}
\end{frame}

\lecturenotes

Заказ на разработку программной продукции, как правило, устанавливает сроки выполнения работ в целом и по отдельным этапам. В нем указываются объемы работ, которые необходимо выполнить к определенным и согласованным с заказчиком контрольным моментам времени. В обязанности менеджера входит распределение указанных работ и определение внутрипроектных контрольных точек. Таким образом, сроки, фиксированные в заказе "--- это общий ресурс проекта, предоставляемый в распоряжение менеджера для распределения.

До тех пор, пока распределение ресурсов для проекта не утверждено, нет смысла говорить о распределении времени выполнения отдельных видов работ. Тем не менее, очень полезно зафиксировать то, каким образом менеджер будет контролировать проектное развитие, как будет направлять его. Иными словами, нужно зафиксировать стратегию планирования. Сведения о ней нужны, в первую очередь, для разработчиков.

Следующие общепризнанные и взаимодополняющие методы, применяемые для распределения временного ресурса и контроля использования времени, можно рекомендовать менеджеру для организации планирования и контроля хода развития проекта: составление календарных планов и ведение графиков сетевого планирования~\cite[с.~48]{Skopin}.

\begin{frame} \frametitle{Календарный план}
  \begin{definition}
    Календарный план "--- это последовательность работ проекта, разбитая по времени на этапы
  \end{definition}
  \begin{table}
  \caption{Пример таблицы с календарным планом}
  \center
  \begin{tabular}{ccccc}
   \hline
   \sz Наименование & \multicolumn{2}{c}{\sz Сроки выполнения\phantom{000}} & \sz Ответственные & \sz Ресурсы \\
   \sz работ & \sz план & \sz факт & \sz исполнители &  \\
   \hline
   1 & 2 & 3 & 4 & 5 \\
   \hline
  \end{tabular}
  \end{table}
\end{frame}

\lecturenotes

Календарный план "--- это поэтапно разбитая и упорядоченная по времени выполнения последовательность работ проекта. Его содержание позволяет руководству планировать деятельность коллектива разработчиков проекта как подразделения фирмы в целом. Как правило, календарный план предъявляется заказчику с тем, чтобы заказчик ориентировался в сроках поэтапного выполнения задания. Это внешние функции календарного плана.

Обычный календарный план представляется в виде таблицы со следующей структурой:

Столбец 1 заполняется в соответствии с разбиением заказанного проекта на составляющие. Обычно глубина рубрикации разбиения зависит от уровня проработанности того или иного фрагмента проекта. По мере углубления декомпозиции и уточнения задач вводятся новые строки таблицы, которые должны вписываться в ранее составленную структуру и не противоречить ограничениям, налагаемым ранее (сроки, исполнители, ресурсы).

Распределение времени и контроль над ним — назначение столбцов 2 и 3. В них указываются календарные даты планируемого (столбец 2) и фактического (столбец 3) сроков выполнения работы, задачи или задания. Планируемое начало работы — это самая ранняя дата, после которой можно приступать к выполнению; конец — это предельный срок отчета исполнителей перед менеджером. Иногда графа планируемых сроков дополняется критическими и целесообразными сроками начала/конца работы. Это позволяет менеджеру более точно следить за распределением временных ресурсов.

Столбец 4 «Ответственный исполнитель и исполнители, роли» задает информацию о том, кто работает над данным заданием, и какая квалификация от исполнителей требуется. Возможно дополнение этого столбца сведениями о том, на какие периоды выделен тот или иной исполнитель для выполнения задания, предполагается ли подмена исполнителей и т.п.

Распределение технических ресурсов и задание сроков их предоставления — содержание столбца 5. Здесь указывается необходимая для выполнения задания техническая, а в ряде случаев, и программная база. Иногда этот раздел дополняется сведениями о лицах, отвечающих за выполнение указываемых требований. Это удобно как для менеджера, так и для ответственных исполнителей: наглядно видны нарушения поставок (несоответствия между плановыми и фактическими сроками)~\cite[с.~49]{Skopin}.

\begin{frame} \frametitle{Достоинства и недостатки календарного плана}
  \begin{block}{Достоинства}
    \begin{itemize}
     \item Легко расширить план новыми рубриками, т.е. строками таблицы
     \item План нагляден
    \end{itemize}
  \end{block}
  \begin{block}{Недостатки}
    \begin{itemize}
     \item Слишком быстро разрастается
     \item Не учитывает загруженность работников
     \item Не подходит для описания параллельных работ
    \end{itemize}
  \end{block}
\end{frame}

\lecturenotes

Дополнение календарного плана новыми рубриками (строками таблицы), в том числе, в процессе выполнения проекта не вызывает трудностей. Другое достоинство календарного плана в том, что он достаточно нагляден.

В то же время, по мере углубления декомпозиции, календарный план имеет тенденцию к разрастанию, а, следовательно, обозревать работы проекта в целом становится все труднее. В результате приходится дублировать логически единый документ, разбивать его на части в соответствии с уровнями ответственности иерархии работников проекта. Другой недостаток календарного плана — его неприспособленность к решению такой важной задачи планирования, как учет загруженности работников и определение текущих потребностей в перераспределении исполнителей. 

Наиболее узким местом календарного плана является то, что его рубрикация зачастую противоречит распараллеливанию работ, привязки параллельных работ и поставок к срокам. Трудно увидеть все нужные показатели на определенный момент времени, трудно решать другие подобные задачи. Для преодоления указанных проблем обычно используют графики сетевого планирования~\cite[с.~50]{Skopin}.

\begin{frame} \frametitle{Сетевое планирование}
  \begin{block}{Графы сетевого планирования}
    \begin{itemize}
     \item Графы, ориентированные на события
     \item Графы зависимостей работ
    \end{itemize}
  \end{block}
\end{frame}

\lecturenotes

Идея всех многочисленных вариантов сетевого планирования заключается в выстраивании работ проекта в виде специальных размеченных графов. Существует два вида таких графов:

Графы, ориентированные на события, вершины которых представляют события, происходящие в ходе развития проекта, определение события дается автором графа. Каждая дуга графа задает работу, начало которой возможно в результате наступления события, представленного начальной вершиной дуги, а выполнение которой является необходимым условием наступления события, представленного конечной вершиной дуги. Совокупность всех входящих в вершину дуг задает необходимое и достаточное условие наступления события, представленного вершиной: завершение выполнения всех работ, определяющих данное событие.

Графы первого вида удобны для всевозможных расчетов обеспеченности работ ресурсами, в том числе и оптимизационных расчетов. В практике планирования развития программных проектов они применяются довольно редко, когда планируются очень большие комплексы работ, для которых сложно обозреть ресурсные потребности в целом. В тех случаях, когда допустимо прямое распределение ресурсов а не его расчет, преимущества событийно ориентированных графов перестают быть существенными. 

Второй вид графов: графы зависимостей работ, вершины которых представляют все работы проекта, а дуги — зависимости работ, определяемые следующим образом. Считается, что, если из одной вершины в другую ведет дуга, то работа, соответствующая второй вершине, может начаться только после завершения первой работы, или вторая работа зависит от первой. Содержательный смысл, вкладываемый в понятие зависимости, может быть различным: от фактической зависимости, когда одна работа использует результаты другой и именно поэтому не может начаться до того, как эти результаты не будут получены, до принудительного упорядочивания работ, например, для учета ресурсных ограничений.

Графы второго вида специально приспособлены для планирования времени и в этом качестве они более универсально применимы. Для сетевого планирования очень больших проектов применяют сочетание событийно-ориентированных графовых описаний проекта и графов зависимостей работ. Имея ввиду проекты, жизненные циклы которых предписываются технологией довольно строго, наиболее характерные для программистской области, в дальнейшем обсуждаются лишь графы зависимостей работ, которые непосредственно связаны с темой данного раздела. 

Описываемая техника применения графов зависимостей соответствует методам, базирующимся на использовании так называемых диаграмм Ганта, которые изображают зависимости в виде, хорошо приспособленном для планирования и отслеживания выполнения работ проекта. Они будут описаны в лекции позднее~\cite[с.~51]{Skopin}.

\subsection{Основные риски, их определение, анализ, распределение приоритетов}

\begin{frame} \frametitle{Оценка рисков}
	\begin{itemize}
		\item Идентификация рисков
		\item Анализ рисков
		\item Распределение приоритетов
	\end{itemize}
\end{frame}

\lecturenotes

Оценка риска состоит из идентификации рисков, их анализа и приоритизации. Идентификация рисков позволяет составить список рисков, которые могут нарушить график проекта. Анализ рисков позволяет оценить вероятность риска и влияние каждого риска, а также риски альтернативных решений по проекту. При расстановке приоритетов создается список рисков, приоритетных по влиянию. Этот
список служит основой для контроля рисков~\cite[с.~81]{McConnell}.

\begin{frame} \frametitle{Контроль рисков}
	\begin{itemize}
		\item Планирование управления рисками
		\item Разрешение рисков
		\item Мониторинг рисков
	\end{itemize}
\end{frame}

\lecturenotes

Контроль рисков состоит из планирования управления рисками, разрешения рисков и мониторинга рисков. Планирование управления рисками подразумевает создание плаан для решения каждого значительного риска. Оно также гарантирует, что планы управления рисками для каждого из отдельных рисков согласуются друг с другом и с общим планом проекта. Разрешение риска подразумевает выполнение плана по работе с каждым значительным риском. Мониторинг рисков "--- это деятельность по мониторингу прогресса в разрешении каждого элемента риска. Мониторинг рисков также может включать текущую деятельность по выявлению новых рисков.

Первым шагом в управлении рисками является определение факторов, которые представляют собой риск для вашего графика. После того, как вы их определили, следующим шагом будет анализ каждого риска для определения его влияния. Вы можете использовать анализ рисков, чтобы выбрать один из нескольких вариантов разработки, или использовать его для управления рисками, связанными с выбранным вами вариантом.

Полезной практикой анализа рисков является определение степени влияния на проект каждого из выявленных рисков. Степень влияния равна вероятности неожиданного убытка, умноженного на размер убытка. Например, если вы считаете, что существует 25-процентный шанс, что потребуется 4 недели дольше, чем ожидалось, чтобы получить одобрение вашего проекта, риск будет составлять 25 процентов, умноженный на 4 недели, что равно 1 неделе. Если вы рассуждаете только о рисках, сбивающих разработку с графика, вы можете выразить все потери в виде недель или месяцев или какой-либо другой единицы времени, что упрощает их сравнение~\cite[с.~82--83]{McConnell}.

\begin{frame} \frametitle{Оценка размеров и вероятностей потерь}
	\begin{block}{Оценка размеров}
		Для оценки сложной потери разбейте её на подзадачи и оценивайте их отдельно, затем объединяйте результат
	\end{block}
	
	\begin{block}{Методы оценки вероятности, уменьшающие её субъективность}
		\begin{itemize}
			\item Оценка специалистом
			\item Групповое обсуждение (метод Дельфи)
		\end{itemize}
	\end{block}
\end{frame}

\lecturenotes

Размер потерь часто легче фиксировать, чем вероятность. Например, вы знаете, что проект будет одобрен либо 1 февраля, либо 1 марта, в зависимости от того, в какой период исполнительный комитет рассматривает предложение по проекту. Если вы предположили, что он будет одобрен 1 февраля, размер риска, что одобрение проекта займет больше времени, чем ожидалось, составит ровно 1 месяц.

В случаях, когда размер потерь трудно оценить напрямую, иногда вы можете разбить его на меньшие потери, оценить их, а затем объединить отдельные оценки в общую оценку. Например, если вы используете три новых инструментария программирования, вы можете оценить потери, возникающие из-за того, что каждый инструмент не дает ожидаемого прироста производительности, а затем эти потери суммировать.

Оценка вероятности потери обычно более субъективна, чем оценка размера потерь, и существует множество практик, доступных для повышения точности этой субъективной оценки. Вот несколько идей:

Попросите человека, хорошо знакомого с системой, оценить вероятность каждого риска, а затем провести анализ оценки риска.

Используйте метод Дельфи. При этом подходе каждый человек оценивает каждый риск индивидуально, а затем обсуждает (в устной или письменной форме) обоснование каждой оценки, особенно очень высокие и низкие. Проведите несколько последовательных раундов, пока оценка не сойдётся~\cite[с.~84]{McConnell}.

\begin{frame} \frametitle{Оценка приоритетов рисков}
	\begin{block}{}
		Первый шаг "--- вычисление степени риска как произведения его вероятности на размер
	\end{block}
	
	\begin{block}{}
		Распределение и перераспределение приоритетов рисков всегда является субъективным
	\end{block}
	
	\begin{block}{}
		Не нужно тратить время на риски с низкой степенью
	\end{block}
\end{frame}

\lecturenotes

После того, как вы создали список рисков расписания, следующим шагом будет определение приоритетов рисков, чтобы вы знали, где сосредоточить свои усилия по управлению рисками. После того, как вы вычислили степень риска, умножив вероятность потери на её размер, отсортируйте риски по этой степени. Это первоначальная, достаточно грубая оценка приоритизации. Вероятно, вам может потребоваться присвоить более высокий или низкий приоритет некоторым рискам.

Обратите внимание, что вероятность и размер потери изначально вычисляются субъективно, поэтому и вычисленный на их основе приоритет будет субъективным. Это ещё одно объяснение грубости первоначальной оценки.

После того, как вы определили элементы с высоким уровнем риска, важно также расставить приоритеты для рисков с низкой степенью. Нет никакого смысла в управлении рисками с низкими потерями, которые также имеют низкую вероятность стать проблемой. Важно не перегрузить себя самой деятельностью по управлению рисками~\cite[с.~95--96]{McConnell}.

\subsection{Преодоление рисков}

\begin{frame} \frametitle{Преодоление рисков}
    \begin{itemize}
     \item Исключение риска
     \item Уменьшение риска
     \item Предупреждение ущерба от риска
     \item Планирование действий в непредвиденных ситуациях
    \end{itemize}
\end{frame}

\lecturenotes

Чтобы снизить влияние рисков на развитие проекта, менеджер должен разработать специальный план. Содержание этого плана — идентификация рисков для данного проекта и мероприятия, снижающие зависимость проекта от рисков.

Преодоление рисков может осуществляться на нескольких уровнях:

Исключение риска. Некоторые рискованные ситуации могут быть исключены полностью. Например, чтобы увольнение работника с ключевой ролью не очень сказалось на продолжении развития проекта, целесообразно с самого начала предложить для занятия этой роли двух человек сравнимой квалификации. В начале проекта их дискуссии полезны для выработки объективных решений, а если один из них откажется от контракта, второй все еще сможет продолжать дело. Полезные дискуссии "--- эта та жертва, которую в ряде случаев возможна для исключения риска. К сожалению, дублирование не может быть рекомендовано для исключения всех рискованных ситуаций.

Уменьшение риска. Если риск не может быть исключен, можно постараться уменьшить его появление на практике. Продолжая пример с увольнением работника, для снижения вероятности этого следует предугадать причины поступка и постараться создать для работника более комфортные условия (повысить заработную плату, создать льготы и т.п.). Нужно, чтобы подобные решения делались не в ответ на заявление об увольнении, а заранее. Это сохранит определенную стабильность в коллективе, которая сама по себе является методом уменьшения риска.

Другой пример уменьшения риска "--- объявление (для заказчика, руководства и коллектива) о пересмотре требований, когда становится понятным, что график выполнения работ может быть сорван. Как и в предыдущем случае, важным моментом здесь является упреждение, т.е. пересмотр требований не в ответ на нарушение графика, а в качестве превентивной меры.

Предупреждение ущерба от риска. Когда не получается удовлетворительно уменьшить риск, следует подготовиться к встрече неприятности так, чтобы минимизировать потери. Если это удается, то продолжение проекта во многих случаях оказывается успешным. В примере с увольнением следует как можно скорее найти замену данному работнику. Естественно, время выполнения проекта увеличится (в частности, потому что новому работнику придется входить в курс дела), но работа все-таки будет продолжена. Это так, но при одном условии: на всех уровнях проектирования заложена возможность отчуждения результатов труда от разработчиков. Если результаты персонифицированы, то трудности подмены для некоторых ролей могут оказаться непреодолимыми.

Планирование действий в непредвиденных ситуациях. Если шаги, направленные на предупреждение ущерба планируются для данного проекта, их выполнение должно быть проведено как можно быстрее. План действий в непредвиденных ситуациях обеспечивает исключение препятствий на этом пути. Так, чтобы произвести скорейшую замену уволенного работника, у менеджера должен быть заранее готов список лиц, способных занять освобождающуюся вакансию. Составной частью этого плана является резервирование в основном графике работ времени, необходимого для вхождения в курс дела нового исполнителя.

Для всех рискованных ситуаций планом управления рисками предусматриваются мероприятия на указанных уровнях в различной комбинации, возможно, дополненные более тонкой реакцией на возникновение риска. Детализация такого плана может быть различной, она зависит от ответственности проекта, его важности для компании и заказчика, с одной стороны, а с другой "--- от степени новизны проекта, отработанности технологии его выполнения.

Очень важно, чтобы исполнители проекта были осведомлены о возможных рисках и о плане управления ими. Это создает уверенность в успехе и подготавливает работников к преодолению трудностей. Для менеджера при составлении плана самое важное "--- не упустить все возможные рискованные ситуации~\cite[с.~68--70]{Skopin}.

\section{Основы планирования работ при управлении проектами}

\subsection{План управления программным проектом}

\begin{frame} \frametitle{План управления программным проектом}
  \begin{block}{Основное содержимое плана}
    \begin{itemize}
     \item Техническое задание
     \item Организация
     \item Процесс
     \item График
     \item Бюджет
    \end{itemize}
  \end{block}
\end{frame}

\lecturenotes

План управления программным проектом является наиболее важным документом проекта. В нем собраны все ранее разработанные документы и спецификации: техническое задание; особенности организации проекта; график и детали процесса выполнения проекта, в том числе план основных фаз и план управления рисками, зависимости и ресурсы проекта; а также бюджет~\cite[с.~252]{Fatrell}.

\begin{frame} \frametitle{Советы по управлению проектом}
    \begin{itemize}
     \item Планируйте выбросить первую версию
     \item Планируйте внесение изменений в систему
     \item Нумеруйте версии продукта
     \item Планируйте организационную структуру для внесения изменений
     \item Планируйте сопровождение системы
    \end{itemize}
\end{frame}

\lecturenotes

Планируйте выбросить первую версию. В то время как программы создаются, тестируются и используются, меняются как фактические потребности пользователя, так и понимание им своих потребностей. Постоянны только изменения, и перепроектирование с новыми идеями неизбежно.

Планируйте внесение изменений в систему. Способы проектирования системы с учетом будущих изменений включают в себя тщательное разбиение на модули, интенсивное использование подпрограмм, точное и полное определение межмодульных интерфейсов и полную их документацию. Очень важно использовать языки высокого уровня и технологии самодокументирования, чтобы уменьшить число ошибок, вызываемых изменений.

Важным приемом является квантование изменений. Каждый продукт должен иметь нумерованные версии, и каждая версия должна иметь свой график работ и дату фиксации, после которой изменения включаются уже в следующую версию.

Планируйте организационную структуру для внесения изменений. Создавать организационную структуру с учетом внесения в будущем изменений значительно труднее, чем проектировать систему с учетом будущих изменений. Каждый получает задание, расширяющее круг его обязанностей, чтобы сделать технически более гибким все подразделение. В больших проектах менеджеру нужно иметь двух или трех высококлассных программистов в качестве резерва, который можно бросить на самый опасный участок боя. Структуру управления также нужно изменять по мере изменения системы. Это означает, что руководитель должен уделить большое внимание тому, чтобы его менеджеры и технический персонал были настолько взаимозаменяемы, насколько позволяют их способности.

Программа не перестает изменяться после своей поставки клиенту. Изменения после поставки называются сопровождением программы. Оно состоит главным образом из изменений, исправляющих конструктивные дефекты. Также эти изменения включают в себя дополнительные функции. Общая стоимость сопровождения широко используемой программы обычно составляет 40 и более процентов стоимости ее разработки. Удивительно, что на стоимость сопровождения сильно влияет число пользователей. Чем больше пользователей, тем больше ошибок они находят~\cite[с.67--70]{Brooks2000}.

\subsection{Определение кадровых ресурсов}

\begin{frame} \frametitle{Кадровые ресурсы}
    
    \begin{block}{Ключевые роли}
      \begin{itemize}
       \item Архитектор проекта
       \item Проектировщики подсистем
       \item Руководители команд разработки подсистем
       \item Специалист по пользовательскому интерфейсу
       \item Эксперт предметной области
      \end{itemize}
    \end{block}
\end{frame}

\lecturenotes

В ходе априорного распределения ресурсов менеджеру полезно определить возможные кандидатуры на ключевые роли коллектива разработчиков с учетом возможности персоналиями совмещать роли или занимать их частично (в течение рабочего дня или в течение определенного календарного периода развития проекта).

К ключевым ролям относятся: архитектор проекта, проектировщики подсистем, руководители команд разработки подсистем, специалист по пользовательскому интерфейсу и эксперт предметной области. 

Кандидаты на ключевые роли — основа коллектива. Заполнение других вакансий определяется, когда такая основа имеется, зачастую даже в ходе развития проекта и по мере необходимости. Поскольку первые участники проекта отвечают только за фазу исследования, для начала проекта минимально необходимыми персоналиями являются те, кто будет выполнять роли архитектора и эксперта предметной области. Но это не значит, что откладывание вопроса об остальных персоналиях — правильное решение. Именно до начала проекта надо определить в общих чертах, с кем в дальнейшем предстоит иметь дело менеджеру.

Задача определения кадровых ресурсов проекта никогда не может быть решена окончательно. Каждый новый работник проекта, каждое увольнение, каждое перераспределение ролей нуждается в корректировке кадровой политики, что проявляется в исправлении графика привлечения сотрудников~\cite[с.~42--46]{Skopin}.

\begin{frame} \frametitle{Операционная бригада по Миллзу}
    \begin{itemize}
     \item Хирург "--- главный программист
     \item Второй пилот "--- проектировщик и программист
     \item Администратор
     \item Редактор документации
     \item Два секретаря
     \item Делопроизводитель
     \item Инструментальщик "--- программист, пишущий внутренние программы
     \item Отладчик "--- тестировщик
     \item Языковед "--- специалист по языку программирования
    \end{itemize}
\end{frame}

\lecturenotes

Хирург. Миллз называет его главным программистом. Он лично определяет технические условия на функциональность и эксплуатационные характеристики программы, проектирует ее, пишет код, отлаживает его и составляет документацию. Он имеет прямой доступ к компьютерной системе, на которой не только производится отладка, но и сохраняются различные версии его программ с возможностью легкой модификации файлов, а также осуществляет редактирование документации. Он должен обладать большим талантом, стажем работы свыше десяти лет и существенными знаниями в системных и прикладных областях, будто прикладная математика, обработка деловых данных или что-либо иное.

Второй пилот. Это второе «я» хирурга, может выполнять любую его работу, но менее опытен. Его главная задача "--- участвовать в проектировании, где он должен думать, обсуждать и оценивать. Хирург испытывает на нем свои идеи, но не связан его предложениями. Часто второй пилот представляет свою бригаду при обсуждении с другими группами функций и интерфейса. Он хорошо знает весь код программы. Он исследует возможности альтернативных стратегий программирования. Он, очевидно, подстраховывает на случай какой-либо беды с хирургом. Он может даже заниматься написанием кода, но не несет ответственности за какую-либо его часть.

Администратор. Хирург "--- начальник, и ему принадлежит последнее слово в отношении персонала, помещений и т.п., но на эти дела он должен тратить как можно меньше времени. Поэтому ему необходим профессиональный администратор, заботой которого будут деньги, люди, помещения, машины, и который будет контактировать с административным механизмом организации в целом.

Редактор. Обязанность разработки документации лежит на хирурге. Чтобы она была максимально понятна, он должен писать ее сам. Это относится к описаниям, предназначенных как для внешнего, так и для внутреннего использования. Задача редактора "--- взять созданный хирургом черновик или запись под диктовку, критически переработать, снабдить ссылками и библиографией, проработать несколько версий и обеспечить публикацию.

Два секретаря. Администратору и редактору нужны секретари. Секретарь администратора обрабатывает переписку, связанную с проектом, а также документы, не относящиеся к продукту.

Делопроизводитель. Он отвечает за регистрацию всех технических данных бригады в библиотеке программного продукта. Особые обязанности, возлагаемые на делопроизводителя, освобождают активных программистов от рутинных работ, систематизируют и обеспечивают надлежащее выполнение тех рутинных операций, которыми часто пренебрегают, и приближают главное, для чего работает команда "--- ее программный продукт.

Инструментальщик. Благодаря возможности в любое время редактировать файлы и тексты и пользоваться службой интерактивной отладки команде редко требуется своя вычислительная машина и группа обслуживающего персонала. Но доступ к этим службам должен осуществляться с безусловной быстротой и надежностью. Только хирург может решать, удовлетворяет ли его работа имеющихся служб. Ему необходим инструментальщик, ответственный за обеспечение доступа к основным службам, а также за создание, поддержку и обновление специальных инструментов — в основном, интерактивных служб, которые требуются его команде. У каждой команды должен быть свой инструментальщик, независимо от качества и надежности имеющихся централизованных служб, и его дело обеспечить всем необходимым или желательным инструментом своего хирурга, а не другие команды. Инструментальщик обычно пишет специализированные утилиты,каталогизированные процедуры, макробиблиотеки.

Отладчик. Хирургу потребуется набор подходящих контрольных примеров для отладки написанных им фрагментов кода, а затем и всей программы. Отладчик является, таким образом, как противником, разрабатывающим контрольные примеры для системного тестирования, исходя из функциональных спецификаций, так и помощником, готовящим данные для ежедневной отладки. Он также обычно планирует последовательность тестирования и создание среды для тестирования компонентов.

Языковед. Вскоре после появления Algol обнаружилось, что в большинстве вычислительных центров есть один-два человека, поражающих своим владением тонкостями языка программирования. Эти эксперты оказываются очень полезными, и с ними часто советуются. Здесь требуется иной талант, чем у хирурга, который является преимущественно системным проектировщиком и мыслит представлениями. Языковед может найти эффективные способы использования языка для решения сложных, неясных и хитроумных задач. Иногда ему требуется провести небольшое исследование (два-три дня) для нахождения удачной технологии.

Таким образом 10 человек могут выполнять хорошо дифференцированные и специализированные роли в команде программистов, организованной по образцу операционной бригады~\cite[с.~23--26]{Brooks2000}.

\subsection{Организация совместной работы}

\begin{frame} \frametitle{Особенности организации работы больших команд}
    \begin{itemize}
     \item Современное проектирование как междисциплинарное согласование
     \item Наличие системного архитектора
     \item Коллективное определение потребностей
    \end{itemize}
\end{frame}

\lecturenotes

С начала ХХ века в организации проектирования произошли два значительных изменения. Во-первых, проектирование стало осуществляться главным образом коллективами, а не отдельными людьми. Во-вторых, в современных коллективах проектировщиков для организации совместной работы все чаще применяются средства передачи данных, а не непосредственное общение.

Рассмотрим особенности организации совместной работы большой команды разработчиков.

Современное проектирование как междисциплинарное согласование

На основе того наблюдения, что в современных условиях достигнута чрезвычайно высокая степень разделения труда по отдельным специальностям, многие ученые пришли к выводу, что в наши дни природа проектирования изменилась: сегодня проектирование должно осуществляться по принципу междисциплинарного согласования в рамках одного коллектива. Из того, что под взаимодействием подразумевается согласование, следует вывод о равноценности мнений всех членов коллектива и о необходимости учитывать мнение каждого. Но это неверно! Если конечная цель состоит в достижении концептуальной целостности, то, руководствуясь необходимостью согласовывать все предложения, считающиеся равноценными, неизбежно приходишь к созданию продукта, перегруженного лишними функциями! Результат такого проектирования можно сравнить с решениями комитета, на заседаниях которого никто не смеет возражать против чьих-либо предложений.

Системный архитектор

Наиболее важным и единственным способом обеспечения концептуальной целостности при коллективном проектировании является наделение полномочиями единовластного системного архитектора. Безусловно, этот человек должен разбираться во всех технологиях, относящихся к проектируемому продукту. Кроме того, он должен обладать опытом проектирования систем такого типа, которая ему поручена. Но важнее всего то, что он должен обладать четким представлением о самой будущей системе и ее назначении и действительно заботиться о достижении концептуальной целостности этой системы.

На протяжении всего процесса проектирования системный архитектор должен выступать в качестве агента, лица, принимающего решения, и защитника интересов будущих пользователей, а также всех прочих людей, для которых создается будущий продукт.

Коллективное определение потребностей и пожеланий со стороны заинтересованных лиц

Если исходить из того, что определение требований к проекту является наиболее сложной из всех задач проектирования, то в этой области увеличение количества привлеченных людей действительно позволяет добиться лучших результатов. Как правило, с привлечением различных людей удается рассмотреть гораздо больше разных вопросов, а сами эти вопросы становятся разнообразнее. С увеличением количества затрагиваемых тем удается лучше понять особенности разрабатываемого проекта. В коллективе должна царить атмосфера сотрудничества, гарантирующая, что каждый член коллектива получит полную возможность найти ответы на все интересующие его вопросы, касающиеся будущего продукта~\cite[с.~94--110]{Brooks2012}

\subsection{Декомпозиция задач}

\begin{frame} \frametitle{Идентификация задач}
	\begin{block}{}
		Источником набора задач может быть личный опыт, опыт компании или данные об аналогичных проектах
	\end{block}
	\begin{block}{}
		Организации SEI, ISO и IEEE выложили в открытый доступ модели жизненного цикла, описания задач, которые выполняются на фазах цикла
	\end{block}
\end{frame}

\lecturenotes

Выбор источников задач зависит от того, существуют ли проекты, предшествующие данному. Если ни вы, ни ваша организация не разрабатывала подобные проекты ранее, либо отсутствуют какие-либо другие наметки, потребуется исследование детализированных рабочих этапов, необходимых для выполнения проекта. Если существует аналогичный проект, предшествующий данному, выполните обзор соответствующих ему действий. В результате вы сможете сформулировать некоторые руководящие указания, применяемые при идентификации действий.

Даже в случае отсутствия предшествующих проектов следует учитывать то, что профессиональные организации, такие как SEI, ISO и IEEE, публикуют стандарты разработки программного обеспечения, на которые можно ориентироваться~\cite[с.~281]{Fatrell}.

\begin{frame} \frametitle{Декомпозиция задач}
	\begin{block}{}
		Задачи разбиваются на подзадачи непосредственно в процессе разработки, вместе с уточнением требований к проекту и обнаружением особенностей продукта или средств разработки
	\end{block}
\end{frame}

\lecturenotes

После разработки требований к продукту производится оценка его размера, а значит и объёма трудозатрат т.е. общего объёма работы, которую необходимо выполнить. Разумеется, не все детали в планах проекта были сформулированы изначально, поскольку они будут уточняться по мере выполнения работ. При этом, по мере уточнения деталей, крупномасштабные задачи разбиваются на более мелкие. Например, в начале осуществления проекта задача может иметь название <<Проектирование интерфейса>>. По мере появления дополнительных сведений относительно требований и особенностей, задача может быть разбита на подзадачи <<Проектирование меню основных функций>> и <<Проектирование меню утилит>>. Разбиение на подзадачи следует проводить до тех пор, пока это будет практичным с точки зрения самой работы и управляемости реализацией проекта~\cite[с.~279--280]{Fatrell}.

\subsection{Вспомогательные инструменты управления проектом}

\begin{frame} \frametitle{Инструменты разработки расписания}
  \begin{itemize}
     \item Диаграмма  Гантта
     \item Диаграмма контрольных событий
     \item Диаграмма по методу критического пути (МКП-диаграмма)
    \end{itemize}
\end{frame}

\lecturenotes

Эти инструменты помогут при составлении расписания проекта с привязкой к календарю. Инструменты разработки расписания применяются вместе с инструментами планирования содержания и стоимости, а результатом их совместного использования является сводный план проекта. Важная роль здесь принадлежит инструментам облегчения процессов, в частности планированию команды, качества и обеспечения, а также продумыванию способов реагирования на риски.

Диаграмма Гантта, или ленточная диаграмма, которая использует горизонтальные полосы для представления операций проекта, показывает даты начала и завершения каждой операции и проекта относительно горизонтальной шкалы времени. Диаграмма Гантта "--- эффективный инструмент для малых и простых проектов, когда нет нужды показывать зависимости между операциями, поскольку они известны всем лицам, участвующим в планировании. По мере увеличения размера и сложности проекта диаграмма постепенно теряет способность справляться со все возрастающим количеством данных, операций и взаимозависимостей между ними.

Диаграмма контрольных событий показывает расположение контрольных событий относительно временной шкалы с целью обозначить ключевые даты и обратить на них внимание руководства. Контрольное событие определяется как момент времени или событие, являющееся кульминационной точкой для многих сходящихся к этой точке зависимостей. Например, «полностью задокументированные требования» могут представлять собой важное контрольное
событие. Традиционно диаграмма контрольных событий служит для того, чтобы обратить внимание руководства на особо важные события, вне зависимости от размера проекта. Как следствие, на диаграмму обычно помещают небольшое число ключевых контрольных событий. Иными словами, диаграмма контрольных событий удобна для представления основных данных о плановом и фактическом ходе продвижения проекта высшему руководству.

Диаграмма по методу критического пути (МКП-диаграмма) "--- это методика начертания сетевой диаграммы для анализа, планирования и составления расписаний проектов. Она предоставляет средства отображения операций проекта в виде узлов или стрелок, определяет, какие из них являются критическими (в смысле влияния на время завершения проекта), и выполняет их календарное планирование так, чтобы достичь целевой даты завершения при минимальной стоимости. Инструмент критического пути изначально был разработан для больших, сложных и кросс-функциональных проектов. Даже в наши дни это основная сфера применения данного метода в силу его способности работать с большим количеством операций и их взаимозависимостей, акцентируя внимание на наиболее критических точках.

Очень хорошо зарекомендовал себя МКП в сочетании с диаграммами Гантта. Извлечение из обширной МКП-диаграммы операций, подлежащих выполнению
в ближайшие недели, представление их в формате диаграммы Гантта, и вручение этих фрагментов диаграммы лицам, ответственным за выполнение соответствующих работ, дает возможность получить и использовать ясные краткосрочные расписания, отражающие ближайшую перспективу~\cite[с.~219--244]{Miloshevich}.

\begin{frame} \frametitle{Инструменты управления содержанием}
  \begin{itemize}
     \item Матрица  координации  изменений
     \item Запрос на внесение изменения в проект
     \item Журнал изменений проекта
    \end{itemize}
\end{frame}

\lecturenotes

Назначение этих инструментов состоит в том, чтобы сосредоточиться на осуществлении контроля за содержанием проекта по ходу исполнения плана проекта. Находясь в тесной координации с инструментами контроля расписания и стоимости, инструменты контроля содержания помогают проектной команде
охватить (взять под контроль) обоснованные изменения проекта и обновленный базовый план содержания. Эта информация об обновлениях используется для ведения отчетности о продвижении проекта и для его закрытия по окончании выполнения. Полезны в этом контексте также инструменты развития команды и инструменты контроля качества, рисков, равно как и другие инструменты контроля.

Следует думать о матрице координации изменений как об удобной «дорожной карте» (плане), позволяющей привести нас к состоянию, описываемому фразой «изменения в проекте контролируются надлежащим образом». В рамках этой цели ССМ помогает четко сформулировать шаги процесса контроля изменений, идентифицировать действия, которые должны быть предприняты, назначать лиц, ответственных за выполнение этих действий, и координировать усилия этих лиц. ССМ должна разрабатываться для каждого проекта, который подлежит (подвергается, может быть подвергнут) изменениям, и по возможности на ранней стадии "--- до определения содержания проекта.

Воздействие изменений, вносимых в проект, на его содержание, расписание, качество и прочие параметры легко может выйти за границы опыта автора данного изменения. Как следствие этого проект может значительно пострадать и в некоторых случаях даже потерпеть крах. По этой причине очень важно обеспечить, чтобы каждое изменение оценивалось дисциплинированным (упорядоченным) и профессиональным образом, прежде чем будет получено разрешение на его практическую реализацию. Запрос на внесение изменений в проект как раз и предназначен для того, чтобы помочь выполнить всестороннее
оценивание предлагаемых изменений. Каждый проект должен использовать такие запросы для просеивания и рационального оценивания предлагаемых изменений. В более крупных проектах использование запросов нуждается в документировании (насколько это возможно) "--- для того, чтобы оставить след в виде зафиксированных изменений.

Изменения в проектах могут не ограничиваться единицами. Напротив, они могут процветать и размножаться. Это создает необходимость записи, нумерации и координации потока изменений и проекте. Подобный мониторинг процесса изменений обеспечивается журналом изменений проекта. В журнале фиксируется каждый запрос на внесение изменения в проект. Когда запрос одобрен, а изменение реализовано, эта информация становится частью журнала. Когда использовать. Журнал должен использоваться в каждом проекте, который подлежит изменениям. Но в случае, если изменения не очень многочисленны, помощь от журнала может быть не очень большой~\cite[с.~453--478]{Miloshevich}.

\begin{frame} \frametitle{Инструменты построения команды}
  \begin{itemize}
     \item Четырехстадийная модель создания проектной команды: формирование, притирка, нормализация и функционирование
     \item Матрица заинтересованных сторон
   \end{itemize}
\end{frame}

\lecturenotes

Эти инструменты фокусируются на человеческой стороне планирования проекта, помогая анализировать ресурсные требования, а также определять, нанимать и организовывать необходимых людей в единую высокопроизводительную проектную команду. Следует добавить, что многие проекты сегодняшнего дня требуют, чтобы процесс построения команды был непрерывным и протекал в течение всего жизненного цикла проекта.

Четырехстадийная модель — это инструмент организации и систематического развития проектных команд. Данная концепция была изначально предложена в 1965 г. B. W. Tuchman. Он идентифицировал 4 стадии, которые должна пройти любая группа в процессе трансформации в единую сплоченную высокопроизводительную рабочую команду: формирование, притирка, нормализация и работа.

Стадия формирования. На этой стадии происходит определение ключевых членов команды и введение их в проект и его миссию. Взаимодействие осуществляется в общем и целом в одном направлении "--- от назначенного лидера проекта, высшего руководства или спонсора проекта к членам зарождающейся команды. Опасения, неразбериха и неясности в описании ролей, что вполне предсказуемо, высоки на этой стадии, в то время как взаимное
доверие, уважение, степень вовлеченности в задачи и приверженность целям проекта, напротив, очень низки. 

Стадия притирки. На этой стадии, называемой также начальной стадией, происходит определение и назначение в проект большого количества членов команды. Члены рабочей группы начинают включаться в работу согласно своему назначению, пытаются понять содержание и требования проекта и распределяют между собой роли и обязанности. Поскольку успех проекта часто зависит от инноваций, кросс-функциональной работы команд, норм исполнения, формируемых членами команды, и принятия решений, команда должна выработать в высшей степени сложный процесс интеграции работ и в значительной степени опираться на способность к самостоятельной постановке целей и самоконтроль.

Стадия нормализации. На этой стадии, называемой также стадией частичного функционирования, большинство членов команды уже назначены и paботают как сплоченная команда над достижением целей проекта. Члены команды начинают чувствовать себя комфортно в исполнении своих ролей и обязанностей и начинают доверять опыту других членов. Команда начинает объединяться в единое целое.

Стадия функционирования. На этой стадии команда представляет собой единое целое. Она подчинена достижению установленных целей проекта. По определению, команда, достигшая стадии функционирования, становится способной к самостоятельной постановке целей.

Матрица заинтересованных сторон "--- это инструмент для выполняемой на систематической основе определения и развития всей проектной команды в целом. Матрица заинтересованных сторон представляет собой двумерную сетку. Вдоль одной стороны матрицы перечислены заинтересованные стороны, а вдоль другой "--- различные факторы, влияющие на успех проекта. Лидеры проекта должны определить фактические и конкретные заинтересованные стороны, факторы успеха и степень их взаимосвязанности, которая указывается в ячейках.

Матрица учитывает сложность проектных сообществ, включающих в себя множество заинтересованных сторон, к которым относятся головные организации, вспомогательные группы, эксперты, вендоры, заказчики, группы с особыми интересами и госорганы. Она обеспечивает каркас для прорисовки
широкого спектра заинтересованных сторон и отражения степени их влияния на успех проекта. Матрица заинтересованных сторон особенно полезна на ранних стадиях формирования команды и планирования команды. Она дает в руки инструмент для идентификации широкого набора ключевых игроков проектной организации, их влияния на различные параметры проекта и критерии успеха~\cite[с.~405--425]{Miloshevich}.

\begin{thebibliography}{99}
\bibitem{Fatrell} Фатрелл Р.Т., Шафер Д.Ф., Шафер Л.И. Управление программными проектами. М.: ИД Вильямс; 2003.
\bibitem{Brooks2012} Брукс Ф. Проектирование процесса проектирования: записки компьютерного эксперта. М.: Вильямс. 2012.
\bibitem{Skopin} Скопин, И. Н. Основы менеджмента программных проектов: курс лекций. 2004.
\bibitem{McConnell} McConnell S. Rapid development: taming wild software schedules. Pearson Education; 1996.
\bibitem{Brooks2000} Брукс Ф. Мифический человеко-месяц или как создаются программные системы. СПб~: Символ-плюс, 2000.
\bibitem{Miloshevich} Милошевич Д.З. Набор инструментов для управления проектами. Компания АйТи. 2008.
\bibitem{Grey} Грей К. Ф., Ларсон Э. У. Управление проектами: Практическое руководство. М.: Издательство «Дело и сервис, 2003.
\end{thebibliography}

\end{document}

%%% Local Variables: 
%%% mode: TeX-pdf
%%% TeX-master: t
%%% End: 
