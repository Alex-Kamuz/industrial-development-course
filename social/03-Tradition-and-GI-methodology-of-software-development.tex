\documentclass{../industrial-development}
\graphicspath{{03-Tradition-and-GI-methodology-of-software-development/}}

\title{Лекция №\,3 по теме «Традиционная и гибкая методологии разработки ПО»}
\author{Хайрулина Мадина, ПИ---21МО}
\date{}

\begin{document}

\begin{frame}
  \titlepage
\end{frame}

\begin{frame}{План лекции}
  \tableofcontents
\end{frame}

%%%%%%%%%%%%%%%%%%%%%%%%%%%%%%%%%%%%%%%%%%страница 1
\section{Традиционное управление разработкой ПО}
\subsection{Предпосылки к появлению традиционных методов}

\begin{frame} \frametitle{Предпосылки к появлению традиционных методов}
	\begin{itemize}
	\item Только около 10\,\% проектов по созданию ПО оказываются успешными, укладываясь в первоначальные бюджетные и временные рамки
	\item Управление определяет успех или неудачу в большей степени, чем технологические преимущества
	\item Количество «выброшенного на свалку» и переделанного ПО является показателем незрелости процесса
	\end{itemize}
\end{frame}

\lecturenotes
Самым лучшим качеством ПО является его гибкость: запрограммировать можно практически все что угодно. Худшим качеством ПО также является его гибкость. Это самое «практически все что угодно» сильно затрудняет планирование, мониторинг и управление разработкой ПО. Ключевые моменты. Традиционный подход к управлению созданием ПО кажется хорошим только в теории; на практике же он основывается на устаревших технологии и методах. Традиционная экономика ПО позволяет получить точку отсчета для традиционных принципов управления созданием ПО. По некоторым моментам имелись разногласия, но основные посылки хорошо согласовывались и дополняли друг друга. Выводы заключались в следующем: 

~\cite{Fowler}.
\begin{thebibliography}{99}
\bibitem{Fowler} \href{https://project.dovidnyk.info/index.php/programnye-proekty/upravlenieproektamiposozdaniyuprogrammnogoobespecheniya/66-tradicionnoe_upravlenie_razrabotkoj_po}{Fowler M. Inversion of Control Containers and the Dependency Injection pattern.}
\end{thebibliography}



%%%%%%%%%%%%%%%%%%%%%%%%%%%%%%%%%%%%%%%%%%страница 2
\subsection{Водопадная модель}
\begin{frame} \frametitle{Водопадная модель}
Три основополагающих момента: 
	\begin{enumerate}
	\item Разработка компьютерных программ состоит из двух основных этапов: анализ и кодирование
	\item Необходимо ввести несколько этапов, включающих в себя определение системных требований, определение требований к программному обеспечению, разработку программы и тестирование
	\item Основной подход, описываемый водопадной моделью, является весьма рискованным и допускает неудачное завершение
	\end{enumerate}
\end{frame}

\lecturenotes
Стадия тестирования, находящаяся в конце цикла разработки, - первый момент, где можно определить реальное время выполнения, объем занимаемой памяти, скорость ввода/вывода и т. д., чтобы сравнить их со значениями, установленными при анализе. Изменения, внесенные в программу, могут оказаться настолько разрушительными, что требования к ПО, на которых основывалась разработка программы, окажутся невыполненными. В таком случае придется либо пересмотреть требования, либо внести существенные изменения в структуру программы. Пункт 1, кажущийся по началу тривиальным, ниже будет развернут в одну из моих основных тем, касающихся управления: отделение стадии разработки от стадии производства. 
~\cite{Fowler}.
\begin{thebibliography}{99}
\bibitem{Fowler} \href{https://project.dovidnyk.info/index.php/programnye-proekty/upravlenieproektamiposozdaniyuprogrammnogoobespecheniya/66-tradicionnoe_upravlenie_razrabotkoj_po}{Fowler M. Inversion of Control Containers and the Dependency Injection pattern.}
\end{thebibliography}


%%%%%%%%%%%%%%%%%%%%%%%%%%%%%%%%%%%%%%%%%%страница 3
\subsection{Условия необходимые для усовершенствования}
\begin{frame} \frametitle{Условия необходимые для усовершенствования}
	\begin{enumerate}
	 \item Завершайте проектирование программы до начала анализа и кодирования 
	 \item Ведите документацию полно и своевременно 
	 \item Выполняйте работу дважды, если это возможно 
	 \item Планируйте, контролируйте и наблюдайте за тестированием 
	 \item Привлекайте к работе заказчика 
	\end{enumerate}
\end{frame}

\lecturenotes
1. Начинайте процесс проектирования с разработчиками программ, а не с аналитиками или программистами. Продумайте, опишите и распределите режимы обработки данных, обращая внимание на риск появления возможных ошибок. 2.Каждый разработчик должен обмениваться информацией с разработчиками интерфейсов, менеджерами и, возможно, заказчиками. На ранних стадиях документация - это и есть проект. Истинная ценность документации заключается в том, что она позволяет поддерживать на более поздних этапах изменения, выполняемые отдельной группой тестирования, отдельной группой сопровождения и группой эксплуатации, которые не знают ПО. 3. Выполняйте работу дважды. Если компьютерная программа разрабатывается впервые, добейтесь того, чтобы версия, которая в конце концов попадет к заказчику для реального использования, была бы на самом деле второй, хотя бы для наиболее критичных проектных/эксплуатационных решений. 4.Задействовать группу специалистов для проведения тестирования, которые не несут никакой ответственности за исходную разработку; выполнить визуальные проверки для обнаружения очевидных ошибок, таких как отсутствие знаков «минус», пропуск множителей двойки, переход по неверному адресу (не следует использовать компьютер для выявления подобного рода ошибок, это слишком накладно); протестировать все логические ветви программы; произвести окончательную проверку на том компьютере, на котором будет применяться программа, а среди этих советов есть несколько хороших и несколько устаревших. Привлечение заказчика к предварительным демонстрациям и к оценке альфа/бета-версий является проверенной и ценной практикой. 
~\cite{Fowler}.

\begin{thebibliography}{99}
\bibitem{Fowler} \href{https://project.dovidnyk.info/index.php/programnye-proekty/upravlenieproektamiposozdaniyuprogrammnogoobespecheniya/66-tradicionnoe_upravlenie_razrabotkoj_po}{Fowler M. Inversion of Control Containers and the Dependency Injection pattern.}
\end{thebibliography}

%%%%%%%%%%%%%%%%%%%%%%%%%%%%%%%%%%%%%%%%%%страница 4
\subsection{Частые проблемы}
\begin{frame} \frametitle{Частые проблемы}
	\begin{itemize}
	  \item Затянувшаяся интеграция и позднее обнаружение ошибок, допущенных при разработке 
		\item Позднее разрешение рисков 
		\item Функциональная декомпозиция, определяемая требованиями 
		\item Противостояние между участниками проекта 
		\item Чрезмерное внимание, уделяемое документации и совещаниям для обмена мнениями 
	\end{itemize}
\end{frame}

\lecturenotes

~\cite{Fowler}.

\begin{thebibliography}{99}
\bibitem{Fowler} \href{https://project.dovidnyk.info/index.php/programnye-proekty/upravlenieproektamiposozdaniyuprogrammnogoobespecheniya/66-tradicionnoe_upravlenie_razrabotkoj_po}{Fowler M. Inversion of Control Containers and the Dependency Injection pattern.}
\end{thebibliography}

%%%%%%%%%%%%%%%%%%%%%%%%%%%%%%%%%%%%%%%%%%страница 5
\begin{frame} \frametitle{Частые проблемы}
	\begin{itemize}
	  \item Затянувшаяся интеграция и позднее обнаружение ошибок, допущенных при разработке 
		\item Позднее разрешение рисков 
		\item Функциональная декомпозиция, определяемая требованиями 
		\item Противостояние между участниками проекта 
		\item Чрезмерное внимание, уделяемое документации и совещаниям для обмена мнениями 
	\end{itemize}
\end{frame}

\lecturenotes
По итогу видим, что разработка традиционными методами, зачастую занимает намного больше времени чем ожидалось, каждое даже самое мелкое изменение влечет за собой очень большую загруженность как для аналитиков, так и для разработчиков, с тестировщиками. Если ошибка будет найдена при тестировании, весь проект переписывается с нуля, что приводит к удорожанию проектов и большим трудозатратам.   
~\cite{Fowler}.

\begin{thebibliography}{99}
\bibitem{Fowler} \href{https://project.dovidnyk.info/index.php/programnye-proekty/upravlenieproektamiposozdaniyuprogrammnogoobespecheniya/66-tradicionnoe_upravlenie_razrabotkoj_po}{Fowler M. Inversion of Control Containers and the Dependency Injection pattern.}
\end{thebibliography}


%%%%%%%%%%%%%%%%%%%%%%%%%%%%%%%%%%%%%%%%%%страница 7
\section{Гибкая методология}
\subsection{Понятие гибкой методологии}
\begin{frame} \frametitle{Понятие гибкой методологии}
  \begin{block}{}
   \alert{Гибкая методология разработки} "--- манифест,определяющий способ мышления и содержащий основные ценности и принципы, на которых базируется несколько подходов к разработке программного обеспечения.
  \end{block}
\end{frame}

\lecturenotes

~\cite{Fowler}.

\begin{thebibliography}{99}
\bibitem{Fowler} \href{https://iiba.ru/scrum-guide-schwaber-and-sutherland/}
\end{thebibliography}


%%%%%%%%%%%%%%%%%%%%%%%%%%%%%%%%%%%%%%%%%%страница 8
\subsection{Методология Agile}
\begin{frame} \frametitle{Методология Agile}
  \begin{block}{}
   \alert{Методология Agile} "--- это семейство процессов разработки, а не единственный подход к разработке программного обеспечения.
  \end{block}
\end{frame}

\lecturenotes

~\cite{Fowler}.

\begin{thebibliography}{99}
\bibitem{Fowler} \href{https://iiba.ru/scrum-guide-schwaber-and-sutherland/}
\end{thebibliography}

%%%%%%%%%%%%%%%%%%%%%%%%%%%%%%%%%%%%%%%%%%страница 9
\subsection{Agile Manifesto}
\begin{frame} \frametitle{Agile Manifesto}
Методологии подписали представители следующих методологий:
	\begin{itemize}
	\item Scrum 
	\item Экстримальное программирование 
	\item DSDM 
	\item Adaptive Software Development 
	\item Crystal Clear 
	\item Feature-Driven Development 
	\item Pragmatic Programming 
	\end{itemize}
\end{frame}

\lecturenotes
Agile Manifesto разработан и принят 11– 13 февраля 2001 года.
Большинство гибких методологий нацелено на минимизацию рисков путем сведения разработки к серии коротких циклов, называемых итерациями, которые обычно длятся одну-две недели. Каждая итерация сама по себе выглядит как программный проект в миниатюре и включает все задачи, необходимые для выдачи миниприроста по функциональности: планирование, анализ требований, проектирование, кодирование, тестирование и документирование. Хотя отдельная итерация, как правило, недостаточна для выпуска новой версии продукта, подразумевается, что гибкий программный проект готов к выпуску в конце каждой итерации. По окончании каждой итерации команда выполняет переоценку приоритетов разработки. 
Agile-методы делают упор на непосредственное общение лицом к лицу. Большинство agile-команд расположены в одном офисе. Как минимум она включает и «заказчиков» (англ. product owner). Это заказчик или его полномочный представитель, определяющий требования к продукту. Эту роль может выполнять менеджер проекта, бизнес-аналитик или клиент. Офис может также включать тестировщиков, дизайнеров интерфейса, технических писателей и менеджеров. 
~\cite{Fowler}.

\begin{thebibliography}{99}
\bibitem{Fowler} \href{https://iiba.ru/scrum-guide-schwaber-and-sutherland/}
\end{thebibliography}

%%%%%%%%%%%%%%%%%%%%%%%%%%%%%%%%%%%%%%%%%%страница 10
\begin{frame} \frametitle{Ценности Agile}
	\begin{itemize}
	\item личности и их взаимодействия важнее, чем процессы и инструменты 
	\item работающее программное обеспечение важнее, чем полная документация 
	\item сотрудничество с заказчиком важнее, чем контрактные обязательства 
	\item реакция на изменения важнее, чем следование плану 
	\end{itemize}
\end{frame}

\lecturenotes
Основным результатом работы по agile- методологии является работающий программный продукт. Расценивая именно работающий программный продукт в качестве единственного показателя работы команды проекта за конечный период времени, создатели концепции agile сформулировали следующие ценности и принципы методологии. 
~\cite{Fowler}.

\begin{thebibliography}{99}
\bibitem{Fowler} \href{https://iiba.ru/scrum-guide-schwaber-and-sutherland/}
\end{thebibliography}

%%%%%%%%%%%%%%%%%%%%%%%%%%%%%%%%%%%%%%%%%%страница 11
\begin{frame} \frametitle{Принципы Agile}
	\begin{itemize}
	\item удовлетворение клиента за счёт ранней и бесперебойной поставки ценного ПО 
	\item приветствие изменения требований, даже в конце разработки (это может повысить конкурентоспособность полученного продукта) 
	\item частая поставка рабочего ПО (каждый месяц или неделю или ещё чаще)
	\item тесное, ежедневное общение заказчика с разработчиками на протяжении всего проекта
	\item проектом занимаются мотивированные личности, которые обеспечены нужными условиями работы, поддержкой и доверием
	\item рекомендуемый метод передачи информации – личный разговор (лицом к лицу)
	\end{itemize}	
\end{frame}

\lecturenotes
~\cite{Fowler}.

\begin{thebibliography}{99}
\bibitem{Fowler} \href{https://iiba.ru/scrum-guide-schwaber-and-sutherland/}
\end{thebibliography}


%%%%%%%%%%%%%%%%%%%%%%%%%%%%%%%%%%%%%%%%%%страница 12
\begin{frame} \frametitle{Принципы Agile}
	\begin{itemize}
	\item работающее ПО – лучший измеритель прогресса 
	\item спонсоры, разработчики и пользователи должны иметь возможность поддерживать постоянный темп на неопределенный срок 
	\item постоянное внимание на улучшение технического мастерства и удобный дизайн 
	\item простота – искусство НЕ делать лишней работы 
	\item лучшие архитектура, требования и дизайн получаются у самоорганизованной команды 
	\item постоянная (постоянная (частая) адаптация (улучшение эффективности работы) к изменяющимся обстоятельствам
	\end{itemize}	
\end{frame}

\lecturenotes
В настоящее время, Scrum является одной из наиболее популярных «методологий» разработки ПО. Согласно определению, Scrum — это каркас разработки, с использованием которого люди могут решать появляющиеся проблемы, при этом продуктивно и производя продукты высочайшей значимости. Когда говорят о методологии Scrum, чаще всего имеют ввиду гибкую методологию разработки ПО, построенную на основе правил и практик Scrum. 
~\cite{Fowler}.

\begin{thebibliography}{99}
\bibitem{Fowler} \href{https://iiba.ru/scrum-guide-schwaber-and-sutherland/}
\end{thebibliography}

%%%%%%%%%%%%%%%%%%%%%%%%%%%%%%%%%%%%%%%%%%страница 13
\subsection{Scrum}
\begin{frame} \frametitle{Методология Scrum}
  \begin{block}{}
   \alert{Методология Scrum} "--- это каркас разработки, с использованием которого люди могут решать появляющиеся проблемы, при этом продуктивно и производя продукты высочайшей значимости.
  \end{block}
\end{frame}

\lecturenotes
В настоящее время, Scrum является одной из наиболее популярных «методологий» разработки ПО. Согласно определению, Scrum — это каркас разработки, с использованием которого люди могут решать появляющиеся проблемы, при этом продуктивно и производя продукты высочайшей значимости. Когда говорят о методологии Scrum, чаще всего имеют ввиду гибкую методологию разработки ПО, построенную на основе правил и практик Scrum. 
~\cite{Fowler}.

\begin{thebibliography}{99}
\bibitem{Fowler} \href{https://iiba.ru/scrum-guide-schwaber-and-sutherland/}
\end{thebibliography}

%%%%%%%%%%%%%%%%%%%%%%%%%%%%%%%%%%%%%%%%%%страница 14
\begin{frame} \frametitle{Роли в Scrum}
В классическом Scrum существует 3 базовых роли: 
\begin{itemize}
	\item Менеджер продукта (Product owner)
	\item Скрам мастер (Scrum master)
	\item Команда разработки (Development team) 
	\end{itemize}	
\end{frame}

\lecturenotes
Product owner (PO) является связующим звеном между командой разработки и заказчиком. Задача PO — максимальное увеличение ценности разрабатываемого продукта 
и работы команды. Одним из основных инструментов PO является Product Backlog. Product Backlog содержит необходимые для выполнения рабочие задачи (такие как Story, Bug, Task и др.), отсортированные в порядке приоритета (срочности). Scrum master (SM) является «служащим лидером» (англ. servant-leader). Задача Scrum Master — помочь команде максимизировать ее эффективность посредством устранения препятствий, помощи, обучении и мотивации команде, помощи PO Команда разработки (Development team, DT) состоит из специалистов, производящих непосредственную работу над производимым продуктом. Согласно The Scrum Guide (документу, являющимся официальным описанием Scrum от его авторов), DT должны обладать следующими качествами и характеристиками: -Быть самоорганизующейся. Никто (включая SM и PO) не может указывать команде каким преобразовать Product Backlog в работающий продукт -Быть многофункциональной, обладать всеми необходимыми навыками для выпуска работающего продукта -За выполняемую работу отвечает вся команда, а не индивидуальные члены команды 
Рекомендуемый размер команды — 7 (плюс-минус 2) человека. Согласно идеологам Scrum, команды большего размера требуют слишком больших ресурсов на коммуникации, в то время как команды меньшего размера повышают риски (за счет возможного отсутствия требуемых навыков) и уменьшают размер работы, который команда может выполнить в единицу времени. 
~\cite{Fowler}.

\begin{thebibliography}{99}
\bibitem{Fowler} \href{https://iiba.ru/scrum-guide-schwaber-and-sutherland/}
\end{thebibliography}

%%%%%%%%%%%%%%%%%%%%%%%%%%%%%%%%%%%%%%%%%%страница 15
\begin{frame} \frametitle{Процесс Scrum}
\begin{itemize}
	\item Sprint
	\item Sprint Planning
	\item Daily Scrum
	\item Sprint Review и Sprint Retrospective
	\end{itemize}	
\end{frame}

\lecturenotes
Основой Scrum является Sprint, в течении которого выполняется работа над продуктом. По окончанию Sprint должна быть получена новая рабочая версия продукта. Sprint всегда ограничен по времени (1-2 недели) и имеет одинаковую продолжительность на протяжении все жизни продукта. Перед началом каждого Sprint производится Sprint Planning, на котором производится оценка содержимого Product Backlog и формирование Sprint Backlog, который содержит задачи (Story, Bugs, Tasks), которые должны быть выполнены в текущем спринте. Каждый спринт должен иметь цель, которая является мотивирующим фактором и достигается с помощью выполнения задач из Sprint Backlog. Каждый день производится Daily Scrum, на котором каждый член команды отвечает на вопросы «что я сделал вчера?», «что я планирую сделать сегодня?», «какие препятствия на своей работе я встретил?». Задача Daily Scrum — определение статуса и прогресса работы над Sprint, раннее обнаружение возникших препятствий, выработка решений по изменению стратегии, необходимых для достижения целей Sprint'а. По окончанию Sprint'а производятся Sprint Review и Sprint Retrospective, задача которых оценить эффективность (производительность) команды в прошедшем Sprint'е, спрогнозировать ожидаемую эффективность (производительность) в следующем спринте, выявлении имеющихся проблем, оценки вероятности завершения всех необходимых работ по продукту и другое.  
~\cite{Fowler}.

\begin{thebibliography}{99}
\bibitem{Fowler} \href{https://iiba.ru/scrum-guide-schwaber-and-sutherland/}
\end{thebibliography}

%%%%%%%%%%%%%%%%%%%%%%%%%%%%%%%%%%%%%%%%%%страница 16
\begin{frame} \frametitle{Важные, часто забываемые особенности}
\begin{enumerate}
	\item Scrum применяется неверно или неполностью 
	\item Недооценена важность работы по обеспечению мотивации команды 
	\item Scrum применяется для продукта, требования к которому противоречат идеологии Scrum
\end{enumerate}
\end{frame}

\lecturenotes
Часто можно услышать, что Scrum не работает, или работает хуже, чем ожидалось. Необходимо заметить, что чаще всего так происходит по одной из следующих причин: 1. Scrum применяется неверно или неполностью. Согласно авторам Scrum, эмпирический опыт является главным источником достоверной информации. Необходимость полного и точного выполнения Scrum указана в The Scrum Guide и обусловлена нетипичной организацией процесса, отсутствием формального лидера и руководителя. 2. Недооценена важность работы по обеспечению мотивации команды. Одним из основных принципов Scrum являются самоорганизующиеся, многофункциональные команды. Согласно исследованиям социологов, численность самомотивированных сотрудников, способных на самоорганизацию не превышает 15 процентов от работоспособного населения. Таким образом, лишь небольшая часть сотрудников способно эффективно работать в Scrum без существенных изменения в ролях Scrum master и Product Owner, что противоречит идеологии Scrum, и потенциально приводит к неверному или неполному использованию Scrum. 3. Scrum применяется для продукта, требования к которому противоречат идеологии Scrum. Scrum относится к семейству Agile, так Scrum приветствует изменения в требованиях в любой момент (Product backlog может быть изменен в любой момент). Это затрудняет использование Scrum в fixed-cost/fixed-time проектах. Идеология Scrum утверждает, что заранее невозможно предусмотреть все изменения, таким образом нет смысла зарание планировать весь проект, ограничившись только just-in-timе планированием, т. е. Планировать только ту работу, которая должна быть выполнена в текущем Sprint. Существуют и иные ограничения. 
~\cite{Fowler}.

\begin{thebibliography}{99}
\bibitem{Fowler} \href{https://iiba.ru/scrum-guide-schwaber-and-sutherland/}
\end{thebibliography}

%%%%%%%%%%%%%%%%%%%%%%%%%%%%%%%%%%%%%%%%%%страница 17
\begin{frame} \frametitle{Итоги Scrum}
Достоинства:
	\begin{enumerate}
		\item Scrum ориентирован на клиента, адаптивен 
		\item Scrum достаточно прост в изучении 
		\item Scrum делает упор на самоорганизующуюся, многофункциональную команду 
	\end{enumerate}	
Недостатки:
	\begin{enumerate}
		\item Scrum задает небольшое количество довольно жестких правил 
		\item Приводит к повышению затрат на отбор персонала, его мотивацию, обучение
	\end{enumerate}	
\end{frame}

\lecturenotes
Scrum обладает достаточно привлекательными достоинствами. Scrum ориентирован на клиента, адаптивен. Scrum дает клиенту возможность делать изменения в требованиях в любой момент времени (но не гарантирует того, что эти изменения будут выполнены). Возможность изменения требований привлекательна для многих заказчиков ПО. Scrum достаточно прост в изучении, позволяет экономить время, за счет исключения не критичных активностей. Scrum позволяет получить потенциально рабочий продукт в конце каждого Sprint'а. Scrum делает упор на самоорганизующуюся, многофункциональную команду, способную решить необходимые задачи с минимальной координацией. Это особенно привлекательно для малых компаний и стартапов, так как избавляет от необходимости от найма или обучения специализированного персонала руководителей. Конечно, у Scrum есть и важные недостатки. Ввиду простоты и минималистичности, Scrum задает небольшое количество довольно жестких правил. Однако это вступает в конфликт с идеей клиентоориентированности в принципе, т. к. клиенту не важны внутренние правила команды разработки, особено если они ограничивают клиента. К примеру, в случае необходимости, по решению клиента Sprint backlog может быть изменен, не смотря на явное противоречие с правилами Scrum. 
Проблема является большей, чем кажется. Т.к. Scrum относится к семейству Agile, в Scrum не принято, к примеру, создание плана коммуникаций и реагирования на риски. Таким образом, делая сложным или невозможным формальное (юридическое или административное) противодействие нарушениям правил Scrum. Другой слабой особенностью Scrum является упор на самоорганизующуюся, многофункциональную команду. При кажущемся снижении затрат на координацию команды, это приводит к повышению затрат на отбор персонала, его мотивацию, обучение. При определенных условиях рынка труда, формирование полноценной, эффективной Scrum команды может быть невозможным. 
~\cite{Fowler}.

\begin{thebibliography}{99}
\bibitem{Fowler} \href{https://iiba.ru/scrum-guide-schwaber-and-sutherland/}
\end{thebibliography}

%%%%%%%%%%%%%%%%%%%%%%%%%%%%%%%%%%%%%%%%%%страница 18
\subsection{DSDM}
\begin{frame} \frametitle{Методология DSDM}
  \begin{block}{}
   \alert{Методология DSDM} "--- это главным образом методика разработки программного обеспечения, основанная на концепции быстрой разработки приложений (Rapid Application Development, RAD), это итеративный и инкрементный подход, который придаёт особое значение продолжительному участию в процессе пользователя/потребителя.
  \end{block}
\end{frame}

\lecturenotes

~\cite{Fowler}.

\begin{thebibliography}{99}
\bibitem{Fowler} \href{https://iiba.ru/scrum-guide-schwaber-and-sutherland/}
\end{thebibliography}

%%%%%%%%%%%%%%%%%%%%%%%%%%%%%%%%%%%%%%%%%%страница 19
\begin{frame} \frametitle{Принципы DSDM}
	\begin{itemize}
	\item Вовлечение пользователя 
	\item Команда должна быть уполномочена принимать важные для проекта решения без согласования с начальством 
	\item Частая поставка версий результата
	\item Главный критерий - как можно более быстрая поставка программного обеспечения
	\end{itemize}	
\end{frame}

\lecturenotes
Вовлечение пользователя - это основа ведения эффективного проекта, где разработчики делят с пользователями рабочее пространство и поэтому принимаемые решения будут более точными
Команда должна быть уполномочена принимать важные для проекта решения без согласования с начальством 
Частая поставка версий результата, с учётом такого правила, что «поставить что-то хорошее раньше - это всегда лучше, чем поставить всё идеально сделанное в конце». Анализ поставок версий с предыдущей итерации учитывается на последующей
Главный критерий - как можно более быстрая поставка программного обеспечения, которое удовлетворяет текущим потребностям рынка. Но в то же время поставка продукта, который удовлетворяет потребностям рынка, менее важна, чем решение критических проблем в функционале продукта 
	\end{itemize}	
~\cite{Fowler}.

\begin{thebibliography}{99}
\bibitem{Fowler} \href{https://iiba.ru/scrum-guide-schwaber-and-sutherland/}
\end{thebibliography}

%%%%%%%%%%%%%%%%%%%%%%%%%%%%%%%%%%%%%%%%%%страница 20
\begin{frame} \frametitle{Основные методики DSDM}
	\begin{itemize}
	\item Тайм-боксинг
	\item MoSCoW 
	\item Прототипирование 
	\item Тестирование 
	\item Рабочая группа
	\item Моделирование 
	\item Управление конфигурацией 
	\end{itemize}	
\end{frame}

\lecturenotes
· Тайм-боксинг 
Тайм-боксинг - одна из основных методик DSDM. Она используется, чтобы достичь главных целей DSDM - разработать информационную систему в сроки, уложиться в бюджет и при этом сохранить качество. Основная идея тайм-боксинга - разделить весь проект на части, каждая со своим бюджетом и сроками выполнения. Для каждой такой части выбираются требования, которые были распределены по принципу MoSCoW. Так как время и бюджет фиксированы, единственное, что можно поменять, - это требования. Так, если проект выбивается из графика или выходит за рамки бюджета, требования с наименьшим приоритетом опускаются. Это не означает, что получится неготовый продукт. Исходя из принципа 20/80 80 процентов проекта получается из 20 процентов требований. Поэтому, как только эти самые важные 20 процентов требований реализованы в системе, она удовлетворяет экономическими требованиям. Стоит заметить, что ни одна система не была идеально построена с первого раза. 
· MoSCoW 
Метод MoSCoW предоставляет путь распределения объектов по приоритетам. В контексте DSDM метод MoSCoW используется для распределения по приоритетам требования. Эта аббревиатура расшифровывается так: 
MUST - требование ДОЛЖНО удовлетворять экономическим нуждам. 
SHOULD - СЛЕДУЕТ ли выполнять это требование, если от него не зависит успех проекта. 
COULD - НУЖНО ли оставить это требование, если оно не действует на деловую потребность проекта. 
WON'T - МОЖНО ли отложить выполнение требования, если ещё есть время. 
· Прототипирование 
Эта методика относится к созданию прототипов системы во время разработки на ранних этапах. Она позволяет выявить недостатки в системе и позволяет будущим пользователям протестировать её. Таким образом реализовано вовлечение пользователя в работу - один из ключевых факторов успеха метода DSDM. 
· Тестирование 
Третья важная сторона достижения цели DSDM - создать информационную систему высокого качества. Чтобы этого добиться, метод DSDM настаивает на проведении тестирования на каждой итерации. Команда проекта вольна сама выбирать способ управления тестированием. 
· Рабочая группа 
Эта одна из методик DSDM, цель которой - собрать вместе различных участников проекта, чтобы обсудить требования, функциональность и наладить взаимопонимание. Участники каждой рабочей группы собираются вместе, чтобы обсудить проект. 
· Моделирование 
Эта методика обязательна и используется с целью визуализировать в виде диаграмм отдельную сторону системы или сферы деятельности, над которыми идёт работа. Моделирование даёт лучшее понимание всей проектной команде сферы деловой активности проекта. 
· Управление конфигурацией 
Хорошая реализация методики управления конфигурацией важна из-за динамической природы DSDM. Так как во время процесса разработки системы происходит множество различных событий и продукты зачастую выпускаются довольно часто, продуктам требуется строгий контроль, чтобы они успешно производились. 

~\cite{Fowler}.

\begin{thebibliography}{99}
\bibitem{Fowler} \href{https://iiba.ru/scrum-guide-schwaber-and-sutherland/}
\end{thebibliography}

%%%%%%%%%%%%%%%%%%%%%%%%%%%%%%%%%%%%%%%%%%страница 21
\begin{frame} \frametitle{Факторы, необходимые для успеха метода DSDM}
	\begin{itemize}
	\item Принятие методики DSDM руководством и всеми работниками
	\item Готовность руководства обеспечить вовлечённость конечных пользователей в работу над проектом
	\item Проектная команда
	\item DSDM выступает за благосклонные отношения между разработчиком и покупателем
	\end{itemize}	
\end{frame}

\lecturenotes
Принятие методики DSDM руководством и всеми работниками. Это обеспечивает мотивацию всех участников с момента запуска проекта и их последующую вовлечённость. 
Готовность руководства обеспечить вовлечённость конечных пользователей в работу над проектом. Процесс прототипирования требует большой вовлечённости пользователей в тестирование и оценивание функциональных прототипов 
Проектная команда. Она должна состоять из опытных членов и в итоге стать постоянным объединением. Важная проблема - доверие и взаимопонимание в проектной команде. Это означает, что команда обладает правом и возможностью принимать важные решения о проекте без формального согласования с руководством, что могло бы отнять много времени. Чтобы команда могла успешно работать над проектом, ей нужны необходимые средства - среда разработки, инструменты для управления проектом и т.д. 
DSDM выступает за благосклонные отношения между разработчиком и покупателем. Это касается проектов, которые разрабатываются внутри самих компаний, а также с использованием сторонних подрядчиков 
Сравнение с другими методами разработки информационных систем 
Уже было разработано и применено в деле множество методов разработки информационных систем. Например, Structured Systems Analysis and Design Method (SSADM), методы быстрой разработки приложений RAD, методы ООП. Большинство этих методов похожи друг на друга и на DSDM. 
Метод экстремального программирования также использует итеративный подход к разработке информационных систем с привлечением пользователей. 
Метод Rational Unified Process - самый похожий на DSDM, также является динамическим методом разработки информационных систем. И опять же в нём применяется итеративный подход к разработке. 

~\cite{Fowler}.

\begin{thebibliography}{99}
\bibitem{Fowler} \href{https://iiba.ru/scrum-guide-schwaber-and-sutherland/}
\end{thebibliography}

%%%%%%%%%%%%%%%%%%%%%%%%%%%%%%%%%%%%%%%%%%страница 22
\begin{frame} \frametitle{Итог по DSDM}
	\begin{itemize}
	\item Он предоставляет необходимые инструменты и способы разработки. Это позволяет пользователям особым образом дополнить процесс разработки своими собственными методами и помочь в принятии решений 
	\item Можно менять не время или средства разработки, а требования к проекту. Такой подход обеспечивает достижение основных целей DSDM - уложиться по времени и не выйти за рамки бюджета 
	\item Взаимопонимание и общение между всеми участниками и их вовлечённость в проект. 
	\end{itemize}	
\end{frame}

\lecturenotes

~\cite{Fowler}.

\begin{thebibliography}{99}
\bibitem{Fowler} \href{https://iiba.ru/scrum-guide-schwaber-and-sutherland/}
\end{thebibliography}


%%%%%%%%%%%%%%%%%%%%%%%%%%%%%%%%%%%%%%%%%%страница 23
\subsection{Экстримальное программирование}
\begin{frame} \frametitle{Методология Экстримального программирования}
  \begin{block}{}
   \alert{Методология Экстримального программирования} "--- это упрощенная методология организации разработки программ для небольших и средних по размеру команд разработчиков, занимающихся созданием программного продукта в условиях неясных или быстро меняющихся требований. 
  \end{block}
\end{frame}

\lecturenotes
Основными целями XP являются повышение доверия заказчика к программному продукту путем предоставления реальных доказательств успешности развития процесса разработки и резкое сокращение сроков разработки продукта. При этом XP сосредоточено на минимизации ошибок на ранних стадиях разработки. Это позволяет добиться максимальной скорости выпуска готового продукта и дает возможность говорить о прогнозируемости работы. Практически все приемы XP направлены на повышение качества программного продукта. 
~\cite{Fowler}.

\begin{thebibliography}{99}
\bibitem{Fowler} \href{https://iiba.ru/scrum-guide-schwaber-and-sutherland/}
\end{thebibliography}

%%%%%%%%%%%%%%%%%%%%%%%%%%%%%%%%%%%%%%%%%%страница 24
\begin{frame} \frametitle{Принципы Экстримального программирования}
\begin{itemize}
	\item Итеративность
	\item Простота решений
	\item Интенсивная разработка малыми группами
	\item Обратная связь с заказчиком 
	\item Достаточная степень смелости и желание идти на риск 
	\end{itemize}	
\end{frame}

\lecturenotes

~\cite{Fowler}.

\begin{thebibliography}{99}
\bibitem{Fowler} \href{https://iiba.ru/scrum-guide-schwaber-and-sutherland/}
\end{thebibliography}

%%%%%%%%%%%%%%%%%%%%%%%%%%%%%%%%%%%%%%%%%%страница 25
\begin{frame} \frametitle{Итеративность}
\begin{itemize}
\item Разработка ведется короткими итерациями
\item За одну итерацию группа программистов обязана реализовать несколько свойств системы
\item Пользовательские истории (ПИ) являются начальной информацией, на основании которой создается модуль
	\end{itemize}	
\end{frame}

\lecturenotes
Описание ПИ короткое – 1-2 абзаца, тогда как ВИ обычно описываются достаточно подробно, с основным и альтернативными потоками, и дополняются моделью. ПИ пишутся самими пользователями, которые в XP являются частью команды, в отличие от ВИ, которые описывает системный аналитик. Отсутствие формализации описания входных данных проекта в XP стремятся компенсировать за счет активного включения в процесс разработки заказчика как полноправного члена команды. 
~\cite{Fowler}.

\begin{thebibliography}{99}
\bibitem{Fowler} \href{https://iiba.ru/scrum-guide-schwaber-and-sutherland/}
\end{thebibliography}

%%%%%%%%%%%%%%%%%%%%%%%%%%%%%%%%%%%%%%%%%%страница 26
\begin{frame} \frametitle{Принципы Экстримального программирования} 
Простота решений
\begin{itemize}
	\item Принимается первое простейшее рабочее решение
	\item Экстремальность метода связана с высокой степенью риска решения
	\item Реализуется минимальный набор главных функций системы на первой и каждой последующей итерации
\end{itemize}	

Интенсивная разработка малыми группами
\begin{itemize}
	\item Все это нацелено на как можно более раннее обнаружение проблем
	\item Парное программирование направлено на решение задачи стабилизации проекта
\end{itemize}	
\end{frame}

\lecturenotes
функциональность расширяется на каждой итерации
~\cite{Fowler}.

\begin{thebibliography}{99}
\bibitem{Fowler} \href{https://iiba.ru/scrum-guide-schwaber-and-sutherland/}
\end{thebibliography}

%%%%%%%%%%%%%%%%%%%%%%%%%%%%%%%%%%%%%%%%%%страница 27
\begin{frame} \frametitle{Интенсивная разработка малыми группами}
\begin{itemize}
\item Все это нацелено на как можно более раннее обнаружение проблем
\item Парное программирование направлено на решение задачи стабилизации проекта
	\end{itemize}	
\end{frame}

\lecturenotes
~\cite{Fowler}.

\begin{thebibliography}{99}
\bibitem{Fowler} \href{https://iiba.ru/scrum-guide-schwaber-and-sutherland/}
\end{thebibliography}
%%%%%%%%%%%%%%%%%%%%%%%%%%%%%%%%%%%%%%%%%%страница 28
\begin{frame} \frametitle{Факторы, необходимые для успеха метода Экстримального программирования}
	\begin{itemize}
		\item Планирование процесса
		\item Тесное взаимодействие с заказчиком
		\item Общесистемные правила именования
		\item Простая архитектура
		\item Рефакторинг
		\item Парное программирование 
		\item 40-часовая рабочая неделя
		\item Коллективное владение кодом
		\item Единые стандарты кодирования
		\item Небольшие релизы
		\item Непрерывная интеграция
		\item Тестирование
	\end{itemize}	
\end{frame}

\lecturenotes
Обычно XP характеризуют набором из 12 правил (практик), которые необходимо выполнять для достижения хорошего результата. Ни одна из практик не является принципиально новой, но в XP они собраны вместе. 
~\cite{Fowler}.

\begin{thebibliography}{99}
\bibitem{Fowler} \href{https://iiba.ru/scrum-guide-schwaber-and-sutherland/}
\end{thebibliography}

%%%%%%%%%%%%%%%%%%%%%%%%%%%%%%%%%%%%%%%%%%страница 29
\begin{frame} \frametitle{Итоги Экстримального программирования}
	\begin{itemize}
		\item Процесс XP является неформальным, но требует высокого уровня самодисциплины 
		\item XP не требует от программистов написания множества отчетов и построения массы моделей 
		\item В XP каждый программист считается квалифицированным работником, который профессионально и с большой ответственностью относится к своим обязанностям
		\item Риск разработки снижается только в команде, которой XP подходит идеально
	\end{itemize}	
\end{frame}

\lecturenotes
Процесс XP является неформальным, но требует высокого уровня самодисциплины. Если это правило не выполняется, то XP мгновенно превращается в хаотичный и неконтролируемый процесс. XP не требует от программистов написания множества отчетов и построения массы моделей. В XP каждый программист считается квалифицированным работником, который профессионально и с большой ответственностью относится к своим обязанностям. Если в команде этого нет, то внедрять XP абсолютно бессмысленно – лучше для начала заняться перестройкой команды. Риск разработки снижается только в команде, которой XP подходит идеально, 
во всех остальных случаях XP – это процесс разработки с наиболее высокой степенью риска, поскольку другие методы снижения коммерческих рисков, кроме человеческого фактора, в XP просто отсутствуют. 
~\cite{Fowler}.

\begin{thebibliography}{99}
\bibitem{Fowler} \href{https://iiba.ru/scrum-guide-schwaber-and-sutherland/}
\end{thebibliography}

%%%%%%%%%%%%%%%%%%%%%%%%%%%%%%%%%%%%%%%%%%страница 30
\subsection{Препятствия для внедрения гибких методологий}
\begin{frame} \frametitle{Препятствия для внедрения гибких методологий}
  \begin{itemize}
		\item Наивный менеджмент ресурсов 
		\item Команды, организованные по функциональной принадлежности  
		\item Команды, организованные по архитектурно-компонентной принадлежности 
		\item Отвлечение внимания 
		\item Нежелание непрерывно пересматривать, приоритезировать и детализировать бэклог продукта 
		\item Технический долг
		\item Нежелание меняться 
	\end{itemize}
\end{frame}

\lecturenotes
Хотя каждая организация находится в своем секторе, использует разные технологии и имеет свою культуру управления, у всех была одна общая болезнь — своего рода «гигантизм». На первый взгляд проблемы организации будут выглядеть как «слишком много задач» или «не достаточно ресурсов», или «нестабильная ситуация на рынке». При ближайшем рассмотрении, ключевые причины окажутся дурными привычками, сформировавшимися «рефлексами» и заблуждениями. Одна из очень известных компаний, которая была примером успешного применения Scrum в 1997 году, обратилась за помощью в Danube Technologies, Inc. в 2009 году, потому что «рынок» показал, что они оказались менее гибкими, чем конкуренты. Начинания по внедрению Scrum, которые начались 1997 году, по-видимому, не могли выдержать десятилетие сосуществования с проблемами, присущими крупным предприятиям. К сожалению, большинство попыток внедрить Scrum в крупных организациях не приводит к долговременным преобразованиям. Препятствия для внедрения Scrum обычно также мешают достижению успеха в бизнесе в целом, а устоявшиеся организации обычно неохотно избавляются от них. 
~\cite{Fowler}.

\begin{thebibliography}{99}
\bibitem{Fowler} \href{https://iiba.ru/scrum-guide-schwaber-and-sutherland/}
\end{thebibliography}

%%%%%%%%%%%%%%%%%%%%%%%%%%%%%%%%%%%%%%%%%%страница 31
\begin{frame} \frametitle{Наивный менеджмент ресурсов }
	Во время спринта члены команды:
	\begin{itemize}
		\item Развивают заинтересованность в общих целях и учатся управлять друг-другом для их достижения
		\item Даже при идеальных условиях команде требуется принять участие в нескольких спринтах, чтобы достичь успеха
\end{itemize}
\end{frame}

\lecturenotes
Наивно рассуждать о людях, как о ресурсах. Новые члены в команде не будут гарантировать увеличение понятия «ресурса команды», и могут даже уменьшить его. После года работы по Scrum один из моих клиентов сказал мне: «После того, как команда сформирована, лучше потерять одного члена команды, чем добавить нового». Совсем другое, если Scrum команда сама пришла к решению о найме нового сотрудника, тогда добавление нового члена команды будет хорошим решением. Но, даже при возможности команды самостоятельно нанимать сотрудников, нецелесообразно увеличивать размер команды больше семи человек. В некоторых случаях увеличение команд может привести к ускорению работы, если это происходит с учетом того, что это нематериальные, неисчисляемые ресурсы. 
~\cite{Fowler}.

\begin{thebibliography}{99}
\bibitem{Fowler} \href{https://iiba.ru/scrum-guide-schwaber-and-sutherland/}
\end{thebibliography}

%%%%%%%%%%%%%%%%%%%%%%%%%%%%%%%%%%%%%%%%%%страница 32
\begin{frame} \frametitle{Команды, организованные по функциональной принадлежности}
		Мы получаем гибкость благодаря:
		\begin{itemize}
		\item Непрерывному анализу требований 
		\item Непрерывному проектированию
		\item Непрерывной интеграции 
		\item Непрерывному тестированию
\end{itemize}
\end{frame}

\lecturenotes
Scrum‐команды это кроссфункциональные команды, которые стараются создать тщательно протестированный инкремент продукта каждый спринт, постепенно наращивая функциональность. Самая популярная книга по Scrum использует фразу “потенциально “Готового” к выпуску Инкремента продукта” (“potentially shippable product increment”) 18 раз. Несмотря на это, я встречаю людей, которые утверждают, что «практикуют Scrum», используя при этом «аналитические спринты» или «спринты проектирования» в начале проекта, откладывая интеграцию и тестирование на потом, и привлекая разные команды для выполнения каждой части работы. Привычки, приобретённые из Водопадной Модели Управления Проектами, скрывают риски до тех пор, пока не становится слишком поздно что-то делать. 
~\cite{Fowler}.

\begin{thebibliography}{99}
\bibitem{Fowler} \href{https://iiba.ru/scrum-guide-schwaber-and-sutherland/}
\end{thebibliography}

%%%%%%%%%%%%%%%%%%%%%%%%%%%%%%%%%%%%%%%%%%страница 33
\begin{frame} \frametitle{Команды, организованные по архитектурно-компонентной принадлежности}
		Такие команды были бы эффективными, если бы разработка нового продукта была бы столь же предсказуемой, как производство.\newline
		На практике:
		\begin{itemize}
		\item Приоритеты меняются, а оценки оказываются неверными 
		\item Трудно привлечь нужных людей в нужное время, чтобы вести разработку 
\end{itemize}
\end{frame}

\lecturenotes
Противоположностью Компонентной Команды является Функциональная Команда, ориентированная на разработку новых возможностей продукта (feature team). Функциональная Команда охватывает как компоненты, так и дисциплины, и, таким образом, способна реализовывать функциональные задачи для бизнеса в тонких, полностью протестированных «вертикальных срезах» продукта. Такой подход призван заменить концепции прошлого века, основываясь на концепциях автономности команд, самоорганизованности и ответственности. Бэклог продукта Функциональных Команд основан на нуждах бизнеса, а не на зависимостях от технологий. Если вы до сих пор используете Диаграммы Гантта, скорее всего у вас пока не организованы Функциональные Команды. Создание Функциональных Команд связано с затратами: разработчики будут должны освоить новые навыки, что замедлит их в начале. К счастью, большинство разработчиков любят осваивать новые навыки. Методы, такие как назначение технического «хранителя компонент» (“component guardian”) для каждой области могут помочь защитить архитектурную целостность, поскольку команды учатся. Как и при любой масштабной разработке, непрерывная интеграция (автоматические тесты, выполняемые гораздо чаще, чем один раз в день) имеет решающее значение для предотвращения необнаруженных дефектов общей работоспособности продукта. После того, как организация научится управлять Функциональными Командами, следующим шагом может быть — создание Функциональных Команд Общего Назначения (general purpose feature teams) с минимальными зависимостями от функциональных областей. Этот процесс может потребовать годы непрерывного обучения внутри организации. 
~\cite{Fowler}.

\begin{thebibliography}{99}
\bibitem{Fowler} \href{https://iiba.ru/scrum-guide-schwaber-and-sutherland/}
\end{thebibliography}

%%%%%%%%%%%%%%%%%%%%%%%%%%%%%%%%%%%%%%%%%%страница 34
\begin{frame} \frametitle{Отвлечение внимания}
		Эффективная горизонтальная коммуникация между Scrum-командами помогает владельцам продуктов спокойно сосредоточиться на обязанностях:
	\begin{itemize}
		\item Определение приоритетов работы 
		\item Определение окончательных требований
\end{itemize}
\end{frame}

\lecturenotes
В типичных крупных организациях тратятся миллионы долларов из-за ненужного переключения между задачами. Командам не хватает сфокусированности и месяцев стабильности в отношении состава команды, чтобы достичь состояния самоорганизации, необходимого для высокой производительности и качества. Некоторые люди постоянно оказываются на нескольких «критических путях» одновременно, что серьезно ограничивает всеобщую производительность. Для эффективного масштабирования эти люди должны стать наставниками, а не исполнителями задач. Чтобы увеличить их влияние, они должны отказаться от некоторого контроля. 
~\cite{Fowler}.

\begin{thebibliography}{99}
\bibitem{Fowler} \href{https://iiba.ru/scrum-guide-schwaber-and-sutherland/}
\end{thebibliography}

%%%%%%%%%%%%%%%%%%%%%%%%%%%%%%%%%%%%%%%%%%страница 35
\begin{frame} \frametitle{Препятствия для внедрения гибких методологий}
Нежелание непрерывно пересматривать, приоритезировать и детализировать бэклог продукта.\newline
Для обнаружения новых задач владелец продукта должен проводить
	\begin{itemize}
		\item Уточняющие встречи (Refinement Meetings) 
		\item Приоретизации и уточнения задач в бэклоге продукта каждый спринт
\end{itemize}
Технический долг
  \begin{itemize}
		\item Технический долг становится причиной проблем в продукте и высокой стоимости изменений. 
	\end{itemize}
\end{frame}

\lecturenotes
 Когда в Бэклоге появляются слишком большие задачи («epics»), необходимо находить их ключевые аспекты и делить их на маленькие подзадачи (обычно называемые «User Stories»). Владелец Продукта управляет объемом работ, решая, какие задачи войдут в Релиз, с учетом данных о скорости разработки и объеме работ, выполняемом в течение Спринта.  
Проблемы управления организацией, в конечном итоге, видны в исходном коде. И «технический долг» становится причиной проблем в продукте и высокой стоимости изменений. В идеале, регрессионные тесты можно автоматизировать с помощью того же языка программирования, что используется при разработке продукта, без использования проприетарных инструментов, которые только усиливают зависимость от специфичных знаний. Навыки, такие как Test Driven Development (TDD), доказали свою ценность в 20-м веке. В 21-м веке от любого разработчика ожидается навыки владение гибкими методологиями. И команды, которые этими навыками еще не обладают, должны пытаться работать с «вертикальными срезами задач», пока они не научатся.  
~\cite{Fowler}.

\begin{thebibliography}{99}
\bibitem{Fowler} \href{https://iiba.ru/scrum-guide-schwaber-and-sutherland/}
\end{thebibliography}

%%%%%%%%%%%%%%%%%%%%%%%%%%%%%%%%%%%%%%%%%%страница 37
\begin{frame} \frametitle{Нежелание меняться}
		Scrum выделяет по одному человеку на команду (ScrumMaster), чтобы уделять основное внимание выявлению и преодолению таких препятствий. \newline
		Это требует:
		\begin{itemize}
			\item Смелости
			\item Воображения 
			\item Поддержки со стороны руководства
		
		\end{itemize}
		Слишком мало Scrum-мастеров уделяют должное внимание таким проблемам, а руководство часто не поддерживает тех, кто это делает. 
\end{frame}

\lecturenotes
~\cite{Fowler}.

\begin{thebibliography}{99}
\bibitem{Fowler} \href{https://iiba.ru/scrum-guide-schwaber-and-sutherland/}
\end{thebibliography}

%%%%%%%%%%%%%%%%%%%%%%%%%%%%%%%%%%%%%%%%%%страница 38
\begin{frame} \frametitle{Вывод}
Люди, ответственные за изменения, должны быть
  \begin{itemize}
		\item Лучше подготовлены к преодолению препятствий для внедрения гибких методологий, осознавая, что ключевые причины проблем в сохранении дурных привычек, сформировавшихся «рефлексах» и заблуждениях
		\end{itemize}
		Оправдания:
		\begin{itemize}
		\item Слишком много задач
		\item Недостаточно ресурсов
		\item Нестабильная ситуация на рынке
		\end{itemize}
		не являются хорошими, чтобы не внедрять гибкие методологии, специально предназначенные для решения этих проблем.

\end{frame}

\lecturenotes
~\cite{Fowler}.

\begin{thebibliography}{99}
\bibitem{Fowler} \href{https://iiba.ru/scrum-guide-schwaber-and-sutherland/}
\end{thebibliography}

\end{document}

%%% Local Variables: 
%%% mode: TeX-pdf
%%% TeX-master: t
%%% End: 
