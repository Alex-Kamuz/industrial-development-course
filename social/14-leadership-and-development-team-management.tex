\documentclass{../industrial-development}
\graphicspath{{14-leadership-and-development-team-management/}}

\title{Лидерство руководителя и управление командой разработчиков}
\author{ ИВТ-21 МО}
\date{}

\begin{document}

\begin{frame}
  \titlepage
\end{frame}

\section{Лидерство руководителя и управление командой разработчиков ПО}

\begin{frame} \frametitle{Понятие лидерства}
 Лидер --- это человек, который может влиять на поведение других людей, брать на себя ответственность, последовательно идти к достижению конкретных целей и вести за собой команду.
\end{frame}

\lecturenotes

Лидерство --- способность оказывать влияние как на~отдельную личность, так и на группы, направляя усилия всех на достижение целей организации. В переводе с английского лидер означает «руководитель», «командир», «глава», «вождь», «ведущий». Группа, решающая значимую проблему, всегда выдвигает для ее решения лидера. Без лидера ни одна группа существовать не может. Лидера можно определить как личность, способную объединять людей ради достижения какой-либо цели. Понятие «лидер» приобретает значение лишь вместе с понятием «цель». Действительно, нелепо бы выглядел лидер, не имеющий цели.
Но иметь цель и достичь ее самостоятельно, в одиночку -- недостаточно, чтобы назваться лидером. Неотъемлемым свойством лидера является наличие хотя бы одного последователя. Роль лидера заключается в умении повести людей за собой, обеспечить существование таких связей между людьми в системе, которые способствовали бы решению конкретных задач в рамках единой цели. Т. е. лидер --- это элемент упорядочивания системы людей. лидерство персонал потенциал.


\begin{frame} \frametitle{Качества лидера}
Тимлидом обычно называют руководителей группы разработчиков в IT-компаниях. Тимлид выполняет следующие функции:

  \begin{itemize}
  \item Техническое ведение проектов
  \item Контроль процессов и минимизация рисков
  \item Контроль рентабельности и мотивация
  \item Контроль своей команды
  \item Контроль качества проекта
  \item Разделение ответственности  
  \end{itemize}
\end{frame}

\lecturenotes

\begin{frame} \frametitle {Техническое лидерство}
Понятие технического лидерства (technical leadership) означает помощь в организации коллективной мыслительной работы по отношению к той или иной технической идее: все участники проекта должны делать одну и ту же систему, а не разные. Технический лидер берет на себя ответственность за выполнение задачи в установленные сроки с требуемым качеством, руководит коллективом разработчиков, отвечает за распределение работ, слаженную работу коллектива
\end{frame}

\begin{frame} \frametitle{Компетенции лидера}
Компетенции, необходимые человеку, претендующему роль лидера команды разработчиков, можно разделить на две группы
  \begin{itemize}
  \item Технические 
  \item Менеджерские 
  \end{itemize}
\end{frame}


\begin{frame} \frametitle{Технические компетенции}
 \begin{itemize}
  \item Тимлид должен уметь преобразовать бизнес-задачу в задачу техническую, понятную для разработчиков. Уметь объяснить разработчикам не только то, что нужно сделать, но и зачем это нужно 
  \item Должен иметь понимание процесса промышленной разработки с точки зрения разделения ответственности в команде, процесса управления требованиями, качеством и технологий, необходимых для обслуживания процесса 
  \item Обладать опытом разработки – от 5 лет и более. Тимлид вдохновляет свою команду личным примером. Но на написание кода тратит не более 20-30\% рабочего времени, так как в приоритете у него управление командой 
  \end{itemize}
\end{frame}
  
  
\begin{frame} \frametitle{Менеджерские компетенции}
 \begin{itemize}
\item Делегирование. Многие тимлиды пишут весь код сами. Тем самым, с одной стороны, они не дают людям ошибаться, учиться на своих ошибках и развиваться. С другой, проваливают свои управленческие задачи
\item Мотивация команды. Те специалисты, которые научатся делать это правильно, получат огромное преимущество перед остальными. Человеку, работающему со своей командой бок о бок гораздо проще мотивировать людей, чем формальному начальнику
\item Наставник для членов своей команды. Необходимо правильно развивать людей — распределять интересные и сложные задачи, правильно давать развивающую обратную связь, составлять планы развития
 \end{itemize}
\end{frame}

\begin{frame} \frametitle{Правила менеджмента программных проектов}

Существует четыре основных правила менеджмента программных проектов:
  \begin{itemize}
  \item Найти нужных людей
  \item Дать им ту работу, для которой они лучше всего подходят
  \item Не забывать о мотивации
  \item Помогать им сплотиться в одну команду 
  \end{itemize}
\end{frame}

\section{Специфика команды разработчиков. Роль лидерства и командности вразработке программного обеспечения}

\begin{frame} \frametitle{Роли в команде разработчиков}
\end{frame}

\begin{frame} \frametitle{Генератор идей}
Оригинальный мыслитель, который дает жизнь новым идеям. Независимый сотрудник с развитым воображением, но подобно остальным людям имеет негативные черты характера — может быть чрезмерно чувствителен к критике. Для успеха генератору идей необходимы конструктивные отношения с руководителем или координатором группы.
\end{frame}


\begin{frame} \frametitle{Исследователь ресурсов}
Так же, как и генератор идей, в состоянии привнести новые идеи в группу, но эти идеи будут заимствованы извне, благодаря широким контактам. Несколько бесцеремонный, гибкий и ищет благоприятные возможности. Обычно разговаривает по телефону или находится где-нибудь на встрече. Не дает развиваться групповой инертности. К отрицательным качествам характера относятся лень, самодовольство и, иногда, требуется кризис или давление обстоятельств, чтобы мотивировать его
\end{frame}

\begin{frame} \frametitle{Координатор}
Обычно формальный лидер группы. Руководит и направляет группу в сторону достижения целей. Может заранее определить, кто из работников хорош для выполнения необходимых задач. Обычно спокойный, уверенный и распорядительный. Однако иногда склонен к излишнему доминированию, и группа становится продолжением его сильного «Я»
\end{frame}

\begin{frame} \frametitle{Мотиватор}
Энергичный и в состоянии внедрять идеи. Видит мир как проект, который требует внедрения. Обычно уверенный, динамичный, эмоциональный и импульсивный. Мотор группы, но может быть раздражительным, несдержанным, нелюбезным
\end{frame}

\begin{frame} \frametitle{Аналитик}
Оценивает предложения и занимает позицию наблюдателя за продвижением. Не дает группе двигаться неправильным путем. Осмотрительный, бесстрастный, имеет аналитический склад ума. Может казаться равнодушным, незаинтересованным, иногда становится чрезмерно критичным
\end{frame}

\begin{frame} \frametitle{Вдохновитель команды}
Стремится объединять и вносить гармонию в отношения между членами группы. Занимает позицию понимающего чужие проблемы, стремится помочь и сглаживает конфликты. По натуре человек добрый, стремится налаживать неформальные отношения. Однако бывает нерешительным в сложных или кризисных ситуациях
\end{frame}

\begin{frame} \frametitle{Реализатор}
Может преобразовать стратегический план в конкретные управленческие задачи, которые доступны для решения. Хороший организатор, методичный и прагматичный. Идентифицируется с группой, лояльный и честный сотрудник. Однако может быть негибким, непреклонным
\end{frame}

\begin{frame} \frametitle{Контролер}
Отлично умеет создавать отчеты о работе группы. Озабочен точным выполнением взятых обязательств и старается не упускать из виду даже мелких деталей. Заставляет придерживаться точного расписания дел, но может становиться излишне тревожным
\end{frame}

\begin{frame} \frametitle{Специалист}
Профессионал, самостоятелен стремится стать экспертом в своей области. Обладает высокой профессиональной/технической экспертизой и знаниями, гордится своей работой. Приносит вклад только в узкой сфере своей профессиональной экспертизы
\end{frame}

\begin{frame} \frametitle{Эффективный лидер}
В списке командных ролей отсутствует роль «лидер». В разное время руководитель команды может и должен брать на себя ту роль, которая позволит сделать команду более сбалансированной. Выступить генератором идей, когда команда стала буксовать в решении проблемы, стать придирчивым критиком-аналитиком, если ощущается переизбыток идей, энтузиазма и оптимизма, или принять на себя роль вдохновителя, когда участники команды приуныли. Способность выполнять разные командные роли, помогать другим людям выполнить их задачи как можно лучше — это одно из обязательных качеств эффективного лидер
\end{frame}


\begin{frame} \frametitle{Эффективный лидер}
Для того чтобы эффективно исполнять эти роли, руководитель обязан обладать следующими профессиональными компетенциями:
\begin{itemize}
\item Видение целей и стратегии их достижения
\item Глубокий анализ проблем и поиск новых возможностей
\item Нацеленность на успех, стремление получить наилучшие результаты
\item Способность сочувствия, понимания состояния участников команды
\item Искренность и открытость в общении
\item Навыки в разрешении конфликтов
\item Умение создавать творческую атмосферу и положительный микроклимат
\item Терпимость, умение принимать людей какие они есть, принятие их права на собственное мнение и на ошибку
\item Умение мотивировать правильное профессиональное поведение членов команды
\item Стремление выявлять и реализовывать индивидуальные возможности для профессионального роста каждого
\item Способность активно "обеспечивать", "доставать", "выбивать" и т.д.
\end{itemize}
\end{frame}

\begin{frame} \frametitle{Типы личности разработчика}
\end{frame}

\begin{frame} \frametitle{Типы личности разработчика}
К подтипу (ощущения) относятся разработчики, живущие в фактическом мире достижимого сегодня. Они конкретны, точны и практичны. Стремятся к специализации. Из них получаются успешные управленцы и настойчивые в реализации программисты. Устанавливают порядок и работают в рамках системы, организуя выполнение задач и завершая их в срок и в рамках бюджета. Люди типа «Инспектор»
\begin{itemize}
\item Любят основательность и детальность, справедливость, практичность. Спокойны, серьезны, настойчивы
\item Характеризуются решительностью в критических ситуациях, являются хранителями традиций
\item Просты и чужды манерности. Они трудолюбивы и упорны в работе
\item Стремятся охватывать все подробности и оперировать точными фактами
\item Могут выполнять и разбираться в сложных и многогранных задачах
\item Не выносят небрежного отношения к ресурсам, все необходимое должно быть в нужное время в нужном месте
\end{itemize}
\end{frame}

\begin{frame} \frametitle{Типы личности разработчика}
К подтипу N (интуиция) относятся разработчики, рассматривающие широкий спектр возможностей, абстрагирующиеся от технических деталей, склонные обобщать и теоретизировать. Из таких разработчиков, на мой взгляд, получаются успешные системные аналитики и архитекторы. Люди типа «аналитик»:

\begin{itemize}
\item Наиболее самоуверенные люди, безгранично верящие в себя и свои силы. Живут своим внутренним миром, концентрируясь на возможностях
\item Интересуются будущим больше, чем прошлым. Это люди, умеющие применить на практике теоретические модели
\item Абсолютно не признают авторитетов, базирующихся на положении, звании или прошлых заслугах. В оценке достоинств какой-либо теории полагаются только на свое собственное мнение, невзирая на то, кто отстаивает сходную позицию
\item Действуют в жизни, как бы участвуя в игре на гигантской шахматной доске, изобретая все новые и новые стратегические и тактические ходы
\item Отстаивают право думать по-своему. Это может и помочь и повредить их карьере так же, как и их способность игнорировать чужие мнения и желания
\item Склонны хвалить за достигнутое и избегают обсуждать негативные черты чьего-либо характера. Они предпочитают движение вперед копанию в недостатках

\end{itemize}
\end{frame}






\section{Тимбилдинг}

\begin{frame} \frametitle{Понятие тимбилдинга}
\begin{block}{}
Тимбилдинг --- созданию благоприятных условий для~работы команды, осуществление мероприятий, нацеленных на сплочение коллектива и его организованности.
\end{block}
\end{frame}

\begin{frame} \frametitle{Задачи тимбилдинга}
Среди задач тимбилдинга можно выделить следующие:
\begin{itemize}
\item знакомство сотрудников разных отделов между собой
\item создание условий для неформального общения
\item повышение эффективности работы команды
\item повышение уровня взаимодействия между сотрудниками
\item сплочение коллектива
\item оценка роли каждого «игрока» в команде: выявление лидеров и аутсайдеров
\item освоение навыков решения нестандартных ситуаций
\item повышение мотивации на достижение коллективных целей
\item снятие стресса, усталости
\item возможность для сотрудников почувствовать себя в новой роли
\end{itemize}
\end{frame}




\begin{thebibliography}{99}

%Информацию собирал с нескольких мест на один слайд, поэтому указал списком, а не метками

С. Архипенков Руководство командой разработчиков разработчиков ПО;
Государственный университет  - Высшая школа экономики Факультет Бизнес-информатики Учебное пособие «Лидерство и управление командой»;
\end{thebibliography}

\end{document}

%%% Local Variables: 
%%% mode: TeX-pdf
%%% TeX-master: t
%%% End: 
