\documentclass{../industrial-development}
\graphicspath{{template/}}

\title{Название лекции}
\author{Иванов Иван Иванович, ИВТ-99 МО}
\date{}

\begin{document}

\begin{frame}
  \titlepage
\end{frame}

\section{Название раздела 1}

\subsection{Название подраздела}

\begin{frame} \frametitle{Пример слайда}
  \begin{block}{Важный факт}
    Промышленная разработка является коллективным процессом\dots
  \end{block}
  
  \begin{itemize}
  \item Тезис 1\dots
  \item Тезис 2\dots
  \item Важный текст на слайдах можно \alert{выделять}\dots
  \end{itemize}
\end{frame}

\lecturenotes

Текст конспекта, относящийся к слайду с указанием источника~\cite[с.~97--99]{Brooks}.

На интернет-источники можно ссылаться не по ГОСТу, но с обязательной гиперссылкой~\cite{Fowler}.

\begin{thebibliography}{99}
\bibitem{Brooks} Брукс Ф. Мифический человеко-месяц или как создаются программные системы. СПб~: Символ-плюс, 2000.
\bibitem{Fowler} \href{https://martinfowler.com/articles/injection.html}{Fowler M. Inversion of Control Containers and the Dependency Injection pattern.}
\end{thebibliography}

\end{document}

%%% Local Variables: 
%%% mode: TeX-pdf
%%% TeX-master: t
%%% End: 
