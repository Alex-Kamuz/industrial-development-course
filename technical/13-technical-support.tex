\documentclass{../industrial-development}
\graphicspath{{13-technical-support/}}

\title{Лекция №\,13 по теме «Организация и автоматизация технической поддержки программных продуктов»}
\author{Хайрулина Мадина, ПИ---21МО}
\date{}

\begin{document}

\begin{frame}
  \titlepage
\end{frame}

\begin{frame}{План лекции}
  \tableofcontents
\end{frame}  

%%%%%%%%%%%%%%%%%%%%%%%%%%%%%%%%%%%%%%%%%%страница 1
\section{Процесс сопровождения}
\subsection{Определение процесса сопровождения}

\begin{frame} \frametitle{Предпосылки к появлению традиционных методов}
	\begin{itemize}
		\alert{Под сопровождением программного обеспечения} 
		 понимают процесс улучшения, оптимизации и устранения дефектов программного обеспечения (ПО) после передачи в эксплуатацию.
	\end{itemize}
\end{frame}

\lecturenotes
Процесс сопровождения является одной из фаз жизненного цикла программного обеспечения, следующей за передачей ПО в эксплуатацию, и завершается выводом его из эксплуатации. В ходе сопровождения в программу вносятся изменения, с тем, чтобы исправить обнаруженные в процессе использования дефекты и недоработки, для добавления новой функциональности, повышения удобства использования (юзабилити) и роста уровня использования ПО. По стандарту ISO/IEC 12207, этот процесс входит в 5 основных процессов жизненного цикла (ЖЦ) ПО: приобретение, поставка, разработка, эксплуатация, сопровождение. 


%%%%%%%%%%%%%%%%%%%%%%%%%%%%%%%%%%%%%%%%%%страница 2
\subsection{Задачи сопровождения}
\begin{frame} \frametitle{Задачи сопровождения}
	\begin{itemize}
		\item устранение сбоев
		\item улучшение дизайна 
		\item расширение функциональных возможностей 
		\item создание интерфейсов взаимодействия с внешними системами 
		\item адаптация для возможности работы на другой аппаратной платформе
		\item миграция унаследованного программного обеспечения
	\end{itemize}
\end{frame}

\lecturenotes


%%%%%%%%%%%%%%%%%%%%%%%%%%%%%%%%%%%%%%%%%%страница 3
\begin{frame} \frametitle{Ожидания пользователей}
	\begin{itemize}
		\item эффективного решения стоящих перед ними задач 
		\item удобного и интуитивно понятного интерфейса 
		\item помощи по всем возникающим вопросам использования ПО 
		\item выполнения их заявок в требуемые сроки 
	\end{itemize}
\end{frame}

\lecturenotes
Именно процесс сопровождения позволяет улучшить удовлетворенность пользователей внедренным ПО. Действительно, общеизвестно, что удовлетворенность пользователей зависит от того, насколько полученный результат соответствует их ожиданиям. 
Все эти задачи можно и нужно выполнять на этапе сопровождения. Кроме того, присущий человечеству консерватизм определяет негативное отношение большинства пользователей к новому ПО. Именно и только стадия сопровождения позволяет примирить с ним пользователей и приучить их с удовольствием и с пользой применять его в своей деятельности. По статистике, удовлетворенность пользователей через год использования ПО в несколько раз выше, чем сразу после внедрения. 
Но, чтобы достичь таких результатов, сопровождение должно осуществляться на должном уровне. Ведь в противном случае эту удовлетворенность можно даже уменьшить. 


%%%%%%%%%%%%%%%%%%%%%%%%%%%%%%%%%%%%%%%%%%страница 4
\subsection{Типы заявок предложений о модификации}
\begin{frame} \frametitle{Типы заявок предложений о модификации}
	\begin{itemize}
		\alert{Корректирующая} --- это реактивное изменение программного продукта для коррекции обнаруженных проблем (после обнаружения).\newline
		\alert{Адаптивная} --- изменение программного продукта после поставки для обеспечения его использования в условиях изменения его (программного продукта) или окружающей среды. 
	\end{itemize}
\end{frame}


%%%%%%%%%%%%%%%%%%%%%%%%%%%%%%%%%%%%%%%%%%страница 5
\begin{frame} \frametitle{Типы заявок предложений о модификации}
	\begin{itemize}
		\alert{Cовершенствующая} --- изменение программного продукта после поставки для улучшения производительности или удобства эксплуатации.\newline
		\alert{Профилактическая} ---  это изменение программного продукта после поставки для выявления и исправления скрытых дефектов в ПО до того, как они станут явными ошибками. 
	\end{itemize}
\end{frame}

\lecturenotes
Следует также отметить, что профилактическое и совершенствующее сопровождение относятся к проактивному подходу к сопровождению, при котором инициатива исходит от обслуживающего персонала, а корректирующее и адаптивное — к реактивному подходу, инициатива которого находится у пользователей. 
Проактивному сопровождению необходимо уделять достаточно внимания, поскольку именно оно в наибольшей степени способствует повышению удовлетворенности пользователей и эффективному развитию программной системы. 

%%%%%%%%%%%%%%%%%%%%%%%%%%%%%%%%%%%%%%%%%%страница 6
\subsection{Концепции сопровождения}
\begin{frame} \frametitle{Концепции сопровождения}
	\begin{enumerate}
	\item Область сопровождения программного средства
		\begin{enumerate} 
		\item Типы выполняемого сопровождения
		\item Сопровождаемый уровень документов
		\item Реакция (чувствительность) на сопровождение (определение ожиданий к сопровождению заказчика)
		\item Обеспечиваемый уровень обучения персонала 
		\item Обеспечение поставки продукта
		\item Организация справочной службы («горячей линии») 
		\end{enumerate}
	\end{enumerate}
\end{frame}

\lecturenotes

%%%%%%%%%%%%%%%%%%%%%%%%%%%%%%%%%%%%%%%%%%страница 7
\begin{frame} \frametitle{Концепции сопровождения}
	\begin{enumerate}[2]
	\item Практическое применение (адаптация) данного процесса
	\end{enumerate}
	\begin{enumerate}[3]
	\item Определение организаций (лиц), ответственных за сопровождение
	\end{enumerate}
	\begin{enumerate}[4]
	\item Оценка стоимости сопровождения: 
		\begin{enumerate}
		\item Проезд до места расположения пользователя 
		\item Обучение как сопроводителей, так и пользователей
		\item СПИ (среда программной инженерии) и СТПС (среда тестирования программного средства) и их ежегодное сопровождение
		\item Персонал (зарплата и премии) 
		\end{enumerate}
	\end{enumerate}
\end{frame}

\lecturenotes
Должен быть сформирован соответствующий план сопровождения. Этот план должен подготавливаться одновременно с разработкой программной системы. План должен определять, как пользователи будут размещать свои запросы на модификацию (изменения) или сообщать об ошибках, сбоях и проблемах. 

%%%%%%%%%%%%%%%%%%%%%%%%%%%%%%%%%%%%%%%%%%страница 8
\begin{frame} \frametitle{Введение}
	\begin{enumerate}
		\item Описание сопровождаемой системы
		\item Определение исходных состояний программного средства 
		\item Описание уровня требуемой поддержки
		\item Определение организаций, осуществляющих сопровождение 
		\item Описание любых условий (протоколов), согласованных между заказчиком и поставщиками 
	\end{enumerate}
\end{frame}

\lecturenotes


%%%%%%%%%%%%%%%%%%%%%%%%%%%%%%%%%%%%%%%%%%страница 9
\begin{frame} \frametitle{Организационные работы и работы по сопровождению}
	\begin{enumerate} \item Роли и обязанности сопроводителя до поставки программного продукта:\end{enumerate}
	\begin{itemize}
	\item Реализация процесса 
	\item Определение инфраструктуры процесса 
	\item Установление процесса обучения 
	\item Установление процесса сопровождения 
	\end{itemize}
\end{frame}

\lecturenotes

%%%%%%%%%%%%%%%%%%%%%%%%%%%%%%%%%%%%%%%%%%страница 10
\begin{frame} \frametitle{Организационные работы и работы по сопровождению}
	\begin{enumerate}[2] \item Роли и обязанности сопроводителя после поставки программного продукта:  \end{enumerate}
	\begin{itemize}
		\item Реализация процесса
		\item Анализы проблем и модификаций (изменений)  
		\item Реализация (внесение) модификаций (изменений) 
		\item Рассмотрение и принятие модификаций (изменений) 
		\item Перенос программного средства в новую среду
	\end{itemize}
\end{frame}

\lecturenotes

%%%%%%%%%%%%%%%%%%%%%%%%%%%%%%%%%%%%%%%%%%страница 11
\begin{frame} \frametitle{Организационные работы и работы по сопровождению}
	\begin{enumerate}[2] \item Роли и обязанности сопроводителя после поставки программного продукта:  \end{enumerate}
	\begin{itemize}
		\item Снятие программного средства с эксплуатации 
		\item Решение проблем (включая справочную службу) 
		\item При необходимости — обучение персонала (сопроводителя и пользователя) 
		\item Усовершенствование процесса
	\end{itemize}
		\begin{enumerate}[3]\item Роль пользователя: \end{enumerate}
	\begin{itemize}
		\item Приемочные испытания  
		\item Взаимосвязи (интерфейсы) с другими организациями
	\end{itemize}
\end{frame}

\lecturenotes


%%%%%%%%%%%%%%%%%%%%%%%%%%%%%%%%%%%%%%%%%%страница 12
\subsection{Ресурсы}
\begin{frame} \frametitle{Ресурсы}
	\begin{enumerate} \item Персонал:  
	\begin{itemize}
		\item Состав персонала для конкретного проекта 
	\end{itemize}
		\item Программные средства: 
		\item Технические средства: 
		\item Оборудование (аппаратура): 
	\begin{itemize}
		\item Определение требований к оборудованию (аппаратуре) системы 
	\end{itemize}
	\end{enumerate}
\end{frame}

\lecturenotes
· состав персонала для конкретного проекта;
Структура, отвечающая за сопровождение, должна проводить общую деятельность по бизнес-планированию, касающуюся бюджетирования, финансового менеджмента и управления человеческими ресурсами в области сопровождения. 
· определение программных средств, необходимых для поддержки эксплуатации системы (с учетом системных требований и требований к СПИ, СТПС и инструментальным средствам); 
· определение требований к оборудованию (аппаратуре) системы (помимо технических средств вычислительной техники); 

%%%%%%%%%%%%%%%%%%%%%%%%%%%%%%%%%%%%%%%%%%страница 13
\begin{frame} \frametitle{Ресурсы}
	\begin{enumerate}[5] \item Документы: \end{enumerate}
	\begin{itemize}
		\item План обеспечения качества
		\item План управления проектом 
		\item План управления конфигурацией 
		\item Документы разработки 
		\item Руководства по сопровождению 
		\item План проведения верификации 
	\end{itemize}
	
\end{frame}

\lecturenotes

%%%%%%%%%%%%%%%%%%%%%%%%%%%%%%%%%%%%%%%%%%страница 14
\begin{frame} \frametitle{Ресурсы}
	\begin{enumerate}[5] \item Документы: \end{enumerate}
	\begin{itemize} 
		\item План проведения аттестации (валидации) 
		\item План тестирования, процедуры тестирования и отчеты о тестировании 
		\item План обучения 
		\item Руководство пользователя
	\end{itemize}
	\begin{enumerate}[6] \item Данные   \end{enumerate}
	\begin{enumerate}[7] \item Другие требования к ресурсам (при необходимости) \end{enumerate}
	
\end{frame}

\lecturenotes

%%%%%%%%%%%%%%%%%%%%%%%%%%%%%%%%%%%%%%%%%%страница 15
\subsection{Выгоды сопровождения}
\begin{frame} \frametitle{Выгоды сопровождения}
	\begin{enumerate} \item Заказчику: 
	\begin{itemize} 
		\item Возможность получить возврат инвестиций на затраты на проект 
		\item Средство ведения бизнеса — необходимый компонент деятельности 
		\item Возможность развиваться 
	\end{itemize}
	\item Вендору : 
	\begin{itemize} 
		\item Возможность эффективно развивать продукт и оперативно исправлять ошибки 
		\item Возможность повысить удовлетворенность партнеров и клиентов 
	\end{itemize}
	\end{enumerate}
\end{frame}

\lecturenotes

%%%%%%%%%%%%%%%%%%%%%%%%%%%%%%%%%%%%%%%%%%страница 16
\begin{frame} \frametitle{Выгоды сопровождения}
	\begin{enumerate}[3] \item Внедренцу --- возможность: \end{enumerate}
	\begin{itemize} 
		\item Продолжения взаимодействия с заказчиком 
		\item Укрепить контакты 
		\item Развиваться 
		\item Сделать работу над ошибками 
		\item Исправить ошибки 
	\end{itemize}
\end{frame}

\lecturenotes
Тем, кто этого еще не сделал, необходимо обратить свое внимание на процесс сопровождения программного обеспечения. 

%%%%%%%%%%%%%%%%%%%%%%%%%%%%%%%%%%%%%%%%%%страница 17
\subsection{Варианты решения проблем}
\begin{frame} \frametitle{Варианты решения проблем}
	\begin{itemize} 
		\item Самостоятельное решение проблемы
		\item Помощь коллег
		\item Обращение за помощью к поставщику сервиса ИТ 
	\end{itemize}
\end{frame}

\lecturenotes
Предметом договоренности в предлагаемой модели выступает некоторая услуга, предоставляемая сотрудникам компании департаментом ИТ, которую в дальнейшем мы будем именовать «сервис ИТ». 
Понятие «сервис ИТ» обычно употребляется в нескольких смыслах. С одной стороны, это услуга, которую служба ИТ предоставляет пользователю. Это может быть, например, доступ в Интернет, или применение автоматизированной системы управления складом для выполнения учетных операций, или использование аналитической системы для проведение маркетингового анализа информации о структуре продаж. С другой стороны, предоставление услуги связано с использованием целого ряда технологий и решений, а возможно, и иных сервисов. В этом смысле всякий сервис ИТ — это структурированный набор оборудования, ПО, технологий, других сервисов (иногда — но вовсе не обязательно — более низкого уровня). Например, предоставление доступа в Интернет предполагает наличие внешнего провайдера, каналов связи, корпоративной сети передачи данных, сервера доступа к корпоративной сети, компьютера с установленным на нем системным ПО и браузером. Эта сторона сервиса ИТ не должна представлять особого интереса для пользователя. Но она очень важна для организации деятельности службы эксплуатации. 
В рамках предлагаемой модели отношений «поставщик сервиса — потребитель сервиса» достижение успеха во многом зависит от действий обеих сторон. Это обуславливается высокой сложностью современных ИС, в том числе клиентского оборудования и ПО (той части сервиса ИТ, которая размещена непосредственно у пользователя). Одной из острых проблем остается обучение пользователей работе с ИС; и поскольку система постоянно развивается, оно должно происходить регулярно. Но обучение ни в коей мере не снизит сложность самой системы, и в процессе работы с ней у пользователя все равно будут возникать различные вопросы и проблемы, связанные с сервисами ИТ, т. е. ему будет нужна помощь. 
В этом случае пользователь может выбрать различные варианты поведения: заняться «самолечением», т. е. попробовать самостоятельно решить проблему; обратиться за помощью к коллегам; обратиться к поставщику сервиса. Каковы плюсы и минусы у каждого из вариантов? 

%%%%%%%%%%%%%%%%%%%%%%%%%%%%%%%%%%%%%%%%%%страница 18
\begin{frame} \frametitle{Варианты решения проблем}
Самостоятельное решение проблемы:
	\begin{itemize} 
		\item Наиболее быстрый способ 
		\item Далеко не все пользователи обладают необходимой квалификацией
		\item Не все проблемы можно решить самостоятельно
	\end{itemize}
\end{frame}

\lecturenotes

%%%%%%%%%%%%%%%%%%%%%%%%%%%%%%%%%%%%%%%%%%страница 20
\begin{frame} \frametitle{Варианты решения проблем}
Помощь коллег:
\begin{itemize} 
		\item Плюс этого способа --- возможность быстро и непосредственно обратиться к кому-то за помощью. 
		\item Минусы --- те же, что и в предыдущем случае. К тому же придется тратить время на объяснение ситуации.
	\end{itemize}
Обращение за помощью к поставщику сервиса ИТ:
	\begin{itemize} 
		\item Наиболее правильный подход
		\item Проблема будет проанализирована со всех сторон и решена оптимальным образом
	\end{itemize}
\end{frame}

\lecturenotes
Основной проблемой при этом обычно становится отсутствие механизма такого обращения. К примеру, пользователю предельно неудобно обращаться за помощью к провайдеру.  
В остальном — только плюсы: имеются квалифицированные специалисты; проблема будет проанализирована со всех сторон и решена оптимальным образом; имеются все необходимые для решения проблемы права; наконец, есть база знаний, которая может помочь в решении проблем. И еще одно неоспоримое преимущество: заключая соглашение с поставщиком сервиса, пользователь тем самым имеет право требовать от него нормальной работы сервиса, в то время как требовать этого от себя или коллег невозможно.

%%%%%%%%%%%%%%%%%%%%%%%%%%%%%%%%%%%%%%%%%%страница 21
\subsection{Выводы}
\begin{frame} \frametitle{Выводы}
\begin{enumerate}  
		\item По всем вопросам, связанным с использованием сервисов ИТ, пользователи должны обращаться только в службу ИТ 
		\item  В рамках службы ИТ должна существовать выделенная группа сотрудников, которые будут отвечать на запросы пользователей 
		\item Порядок оказания поддержки пользователям должен быть четко формализован для всех участников процесса: пользователей, диспетчеров, специалистов информационной службы и внешних поставщиков сервисов 
\end{enumerate}  
\end{frame}

\lecturenotes
С формально-организационной точки зрения эти принципы позволяют построить реально полезную для организации систему поддержки пользователей. Для дальнейшего обсуждения проблемы определим более точно предмет деятельности службы поддержки — какого типа помощь может понадобиться пользователям. 

%%%%%%%%%%%%%%%%%%%%%%%%%%%%%%%%%%%%%%%%%%страница 22
\section{Обращения пользователей}
\subsection{Классификация}
\begin{frame} \frametitle{Классификация}
	\begin{itemize} 
		\item Запрос обслуживания
		\item Запрос информации (консультации)
		\item Инцидент. Пользователь не может нормально работать: сервис ИТ недоступен или качество сервиса его удовлетворяет
		\item Запрос документации
		\item Запрос на внесение изменений 
	\end{itemize}
\end{frame}

\lecturenotes
Такая классификация позволяет более эффективно построить обработку обращений. В зависимости от типа обращения оператор диспетчерской службы принимает решение о том, кто будет исполнителем. Из разных типов обращений для бизнеса по понятным соображениям наиболее критичен инцидент. Обработка инцидента — основной процесс, выполняемый службой поддержки пользователей. Сюда входит выполнение мероприятий по восстановлению работоспособности сервиса ИТ в кратчайшие сроки. В некоторых организациях приоритет данной задачи настолько выше, чем у остальных, что деятельность службы поддержки пользователей практически полностью сводится к устранению инцидентов. 

%%%%%%%%%%%%%%%%%%%%%%%%%%%%%%%%%%%%%%%%%%страница 23
\subsection{Правила взаимодействия}
\begin{frame} \frametitle{Правила взаимодействия}
	Взаимодействие пользователей со службой ИТ:
	\begin{itemize} 
		\item Каким образом пользователь может обратиться за поддержкой 
		\item Какие запросы пользователя должны обрабатываться службой ИТ
		\item Как пользователю уточнить текущее состояние обработки своего запроса
		\item Как пользователь подтверждает закрытие своего запроса
	\end{itemize}
\end{frame}

\lecturenotes

%%%%%%%%%%%%%%%%%%%%%%%%%%%%%%%%%%%%%%%%%страница 24
\begin{frame} \frametitle{Правила взаимодействия}
	Работа сотрудников службы поддержки пользователей:
	\begin{itemize} 
		\item Как учитываются обращения пользователей 
		\item Как определяются приоритеты обработки обращений 
		\item Как определяются исполнители, ответственные за обработку обращений 
		\item Каким образом специалист, ответственный за обработку обращения, осуществляет эту обработку 
		\item Как сотрудники службы поддержки учитывают выполненные действия и отчитываются о проделанной работе 
	\end{itemize}
\end{frame}

\lecturenotes

%%%%%%%%%%%%%%%%%%%%%%%%%%%%%%%%%%%%%%%%%страница 25
\begin{frame} \frametitle{Правила взаимодействия}
Взаимодействие сотрудников службы ИТ между собой: 
	\begin{itemize} 
		\item Каким образом специалист назначается ответственным за обработку обращения пользователя 
		\item Как контролируется процесс обработки обращения 
		\item Как назначить другого ответственного за обработку обращения 
	\end{itemize}
\end{frame}

\lecturenotes




\end{document}

%%% Local Variables: 
%%% mode: TeX-pdf
%%% TeX-master: t
%%% End: 
